在評估 unit encode 系統的效果時,我們關注的重點是兩個方面:
1. 每個 unit 所代表的 phn 越集中越好。
2. 每個 phn 由越少的 unit 代表越好,這樣看到某個 unit 時,幾乎可以確定它代表哪個 phn。

這兩個指標可以通過圖上的顏色深淺來直觀地解釋兩個 purity 的意義:
* phn purity 是每個 column 取最大值後相加。
* unit purity 則是每個 row 取最大值後的 pxy 總和。

可以看出,unit purity 受限於最多只能取 41 個 pxy 的總和。而當 cluster 數量增加時,phn purity 能取到的 pxy 總和也會增加。當 cluster 數量與 frame 數量相同時,phn purity 可達到 100\%。

另一方面,隨著 cluster 數量增加,每個格子的 pxy 會因為 unit 數量增加而被稀釋,但 unit purity 的 pxy 總和會被 phn 數量限制。因此,cluster 增加時,unit purity 會降低,這一點也可以觀察到。

%%%%%%%%%%%%%%%


        
        、HuBERT 模型分 100 群所得離散單元之範例。給予一段音檔的音位轉寫與每個音位對應的起始與終止時間,以及離散單元的序列,我們可以將離散單元依照語音表徵的時間解析度,找出每個語音表徵向量對應的語音訊號時間區段,對應到離散單元序列上;同時對於音位的時間段對應到音框編號上,以獲得在音框上對齊的音位序列,隨後進行統計分析。