
\section{語音學分類(Phone Type)}

  
除了單一音位本身的特性以外,由於音位之間存在相似的特徵,可以分成幾個組別。
依照 \cite{10097097, abdullah23_interspeech} 的分組方式,對英語的音位進行分類並合併比對數據,
觀察這些離散單元是否有擷取到相似的發聲特徵,而不單純只是把音位分成約 50 類完全獨立的標籤。以下簡介六個音位組別:



\begin{itemize}
    \item 塞音(Plosive):以完全阻塞氣流的方式發音的音位,包含 /p/、/b/、/t/、/d/、/k/、/g/ 六種
    \item 擦音(Fricative):藉由在口腔中形成的縫隙,使氣流通過時摩擦形成的發音,包含 /f/、/v/、/s/、/z/、/\textesh/ (sh)、/\textyogh/ (如「garage」的 「-ge」)、/θ/ (無聲的 th)、/ð/ (有聲的 th)、/h/ 九種
    \item 塞擦音(Affricate):由塞音和同部位的擦音同時發出的輔音,英語中只有 /t\textesh/ 和 /d\textyogh/ 兩種,即 ch 和 j 的發音
    \item 鼻音(Nasal):使氣流通過鼻腔形成的聲音,有 /m/、/n/、/ŋ/ (ng) 三種
    \item 近音(Approximant):又稱半元音,為介於元音和輔音之間的聲音,有 /j/ (為 y 作為輔音時的發音)、/r/、/l/、/w/
    \item 元音(Vowel):可自成音節的音位,包含發音位置固定的單元音(Monophthong)和會移動的的雙元音(Diphthong),通常以 a、e、i、o、u 字母產生的聲音皆屬於此類別
\end{itemize}


透過將音位分組後,形成新的語音標註,並重新分析統計指標,觀察在純度等數據是否顯示離散單元與語音的發音方式具有更明確的關聯性。

