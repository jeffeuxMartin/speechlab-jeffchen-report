
\section{衡量方式}


% 首先是
本次研究主要探討純度(Purity)、熵(Entropy)和相互資訊(Mutual Information,MI)等指標,
這些標準在 HuBERT 中採用 \cite{hsu_hubert_2021, hsu_hubert_2021-2},用於比對機器學習過程中得到的偽標註與人類標註之間的相關性(Correlation)。以下對各標準進行詳細解釋:

不論是何種語音基石模型,
語音表徵的基本單位是音框(Frame)。因此一段語句(Utterance)的語音離散單元被表示為 $[y_1, \cdots\cdots, y_T]$。其中 $T$ 是該段語句的音框總數。對於該段語句,若給予一段在音框上對齊的語音學標註(Phonetic Label) $[z_1, \cdots\cdots, z_T]$,此時我們可以將離散單元與標註之間配對的出現次數,寫為一個雙變數的共同分佈(Joint Distribution)
\begin{align}
    p_{yz} = \frac{\sum^T_{t=1}[{y_t = i \wedge z_t = j}]}{T}
\end{align}

其中 $i$ 是第 $i$ 個音位類別,而 $j$ 指編號為$j$的離散單元。兩個變數的邊際機率(Marginal Probability)分別為
\begin{align}
    p_z(j) & =\sum_i{p_{yz}(i, j)} \\
    p_y(i) & =\sum_j{p_{yz}(i, j)}
\end{align}

因此,對於每一個音位 $i$ 而言,這個音位對應最可能的離散單元為
\begin{align}
    z^\ast(i) = \arg\max_j p_{yz}(i, j)
\end{align}
與之相對應的,對於每一個離散單元的類別 $j$ 則可以找到機率最高的音位
\begin{align}
    y^\ast(j) = \arg\max_i p_{yz}(i,j)
\end{align}

% 於是我們可以計算出以下指標:
透過這些定義,以下分節介紹將要分析的指標:

\subsection{純度}
  
本指標考慮音位和離散單元兩個序列之間對應的最高機率,因此從音位與離散單元的角度出發,可以得到以下兩項數據:

\paragraph{音位純度(Phoneme Purity)}

考慮每個離散單元對應的音位中,最高機率音位的機率,表示為
\begin{align}
    \mathbb{E}_{p_z(j)}\left[p_{y|z}(y^*(j)|j) \right]
\end{align}

此指標表示該單元是否對其對應的音位有足夠的代表性。

\paragraph{分群純度(Cluster Purity)}

與音位純度相對,改以每個音位的角度,考慮對應單元類別的機率
\begin{align}
    \mathbb{E}_{p_y(i)}\left[p_{z|y}(z^*(i)|i) \right]
\end{align}

由於離散表徵進行分群演算法時的類別數是一項超參數(Hyperparameter),且通常離散單元的分群數量會比音位多,因此該統計數據本身不直接具有語音學的解釋意義,而且在分群數量很多時會顯著下降。
然而該指標在考量音位純度時必須一併考慮,
因為當分群數非常多時,分群純度過低
暗示離散單元做不到歸納音位類別的效果,
使得音位純度失去其意義。一個極端的情形是每一個音框都給予不同的離散單元編號,如此音位純度可以達到
100\%。

\subsection{熵和相互資訊}
  
除了純度提供「最高機率」的對應關係,根據 HuBERT 論文 \cite{hsu_hubert_2021-2} 中的分析方式,我們也可以從資訊理論的角度,觀察兩個序列的熵和相互資訊。

\paragraph{熵(Entropy)}

熵的定義按照資訊理論,衡量兩個序列中標籤類別出現機率的不確定性(Uncertainty),公式寫作:
\begin{align}
    H(y) & = \sum_i{p_y(i)\log p_y(i)} \\
    H(z) & = \sum_j{p_z(j)\log p_z(j)}
\end{align}
其中 $H(y)$ 和 $H(z)$ 分別為音位和離散單元的熵,數值愈高分別表示各種音位和離散單元出現的機率愈平均。

\paragraph{以音位標準化之相互資訊(Phone-normalized Mutual Information,PNMI)}

本數據以「觀察到某一個離散單元,能降低多少音位標註的不確定性」,定義該離散單元的出現背後提供了多少音位的資訊。公式寫為:
\begin{align}
    \frac{I(y;z)}{H(y)} & =\cfrac{\sum_i \sum_j p_{yz}(i, j) \log \cfrac{p_{yz}(i, j)}{p_y(i)p_z(j)}}{\sum_i p_y(i) \log p_y(i)} \\
                        & =\frac{H(y)-H(y|z)}{H(y)}                                                                              \\
                        & =1-\frac{H(y|z)}{H(y)}
\end{align}

該項數據愈高,表示離散單元的分群愈能提供語音音位的資訊,是一個品質更好的分群結果。由於離散單元能多好的對應到音位才是人們所關心的問題,因此與純度不同,只以音位的角度出發,而不考慮以離散單元分群的角度。

% \input{3b}
