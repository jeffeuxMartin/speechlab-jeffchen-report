   %%%%%%%%%%%%%%%%%%%%%%%%%  
\mychcnt{3}
\chapter{單一語音離散表徵與音位的關係}

\input{chapters/ch3-pre}
% \mysccnt{5}
\section{分析方式}

  針對模型得出之離散單元與音位標註之間的對應關係,為了更直觀的解釋這些指標的意義,並且看清楚這些數字背後之間代表的現象與細部特徵,我們將音位與離散單元的共同機率分佈 \(p_{yz}\) 用熱圖(Heatmap)呈現,來解釋這些指標的意義,並以此進一步往下深入探討。

        首先,這裡以 HuBERT 為基石模型、離散單元分群數為 50 的統計數據為例,圖 \ref{fig:hubert-50-joint-byprob} 說明我們如何分析語音離散表徵與音位標註的關係。

\begin{figure}
    \centering
    \includegraphics[width=1\linewidth]{figures/hubert-50-joint-byprob.png}
    \caption{HuBERT 模型、分群數為 50 之離散單元與音位標註的共同機率分佈圖}
    \label{fig:hubert-50-joint-byprob}
\end{figure}

        圖中的縱軸表示各個音位,橫軸表示各個離散單元。在這張圖中,縱軸的音位是按照其邊際機率 \(p_y(i)\) 由高至低排序;橫軸的離散單元則是依據其對應的最高機率音位 \(y^\ast(j)\) 的縱軸排序位置進行排列。\footnote{如果兩個離散單元 \(j_1\) 和 \(j_2\) 對應到相同的音位 \(y^\ast = y^\ast(j_1) = y^\ast(j_2)\),則依照機率值 \(p_{yz}(y^\ast, j_1)\) 和 \(p_{yz}(y^\ast, j_2)\) 由高到低進行排序,對於多個離散單元的情況以此類推。} 這樣可以在熱圖上呈現出由左上至右下的對應關係。

        藉由熱圖的幫助,我們不僅可以更加完整清晰的觀察離散單元與音位標註之間的關係,對於純度大小的意義也可以從此圖上有更具體的了解:

        \begin{enumerate}
            \item 將每個直行(Column)取最大值相加後的總和即為音位純度。如果每個離散單元與音位都相對集中,則可以得到較高的音位純度。且如同指標說明的小節所述,當分群數量增加時,音位純度能夠在每個直行上取到更多的機率值總和。最極致的情況是,當分群數量與音框數量相同,音位純度可以達到 100\%。這些性質透過熱圖的視覺化呈現,可以被更直觀的說明。

            \item 將每個橫列(Row)取最大值相加後的總和則是分群純度。如果每個音位都也恰好可以很集中的對應到少數幾個離散單元,則此數值將較高,每個橫列最高可以貢獻的值為該音位出現的機率 $p_{y}(i)$。同樣的,當分群數量增加時,隨著直行數目的增多,單看每一個音位對應的橫列,會發現每個格子的機率值隨之被稀釋。受到音位標註類別數的限制,分群純度最高只能取 41 個 $p_{yz}$ 值的總和,使得單位純度因而明顯降低。
        \end{enumerate}

        此外,比起只有音位與分群純度兩個數字,機率熱圖不但可以呈現純度指標的綜觀解釋性意義,我們還可以分門別類對個別的音位與離散單元進行細部探討。畢竟,模型的虛擬標註與實際人類給予的標註資料並不能總是完美而集中的互相對應。我們想知道的細部觀察可分為兩個層面:

        \begin{enumerate}
            \item 從離散單元的角度出發,每個單元 $j$ 所對應的音位是如何的集中,因而多能夠代表這個單元中最高機率的音位 $y^*(j)$?如果恰巧該單元對應的音位條件機率分佈 $p_{y|z}(i|j)$ 較為分散,那與這個單元最相關,也就是條件機率前幾高的音位之間,又是否呈現特定關係?
            \item 反之從音素標註考慮,對於每個音位 $i$,觀察它所對應的離散單元集中程度,也就是離散單元條件機率分佈 $p_{z|y}(j|i)$ 得出的熵值 $H(z|y)$ ,可否觀察到特定一些音位較難或較易被離散單元集中歸類,進而推論模型是否善於辨認該音位的發音特性。
        \end{enumerate}

        因此,接下來我們會先以綜觀的角度,比較來自不同語音表徵與分群數的離散單元,它們的純度、相互資訊等數據,並輔以對應的機率熱圖佐證,觀察離散表徵在捕捉發音資訊的能力強弱。此後,分別從離散單元和音位兩個面向,藉助音位分類知識的幫助,進行細部觀察。最後將細部觀察的結論,重新對應回機率熱圖上的深淺規律,以對這些觀察的進行驗證。

\section{分析結果}

\subsection{綜觀分析}

  表 \ref{tab:single-cluster-results} 提供了不同語音表徵與分群數的純度和相互資訊的指標數據。

        首先,我們先比較同樣是分群數為 50 時,四種語音表徵的聯合分佈熱圖,呈現在圖 \ref{fig:ch3-heatmap-model-comparison--part1} 中。直接觀察聯合分佈熱圖,我們可以明顯觀察到,HuBERT 和 CPC 在熱圖上具備較多較深且清晰的方塊,也就是音位與離散單元之間的對應相對 Wav2vec 2.0 與 LogMel 更為明確,反映出 HuBERT 和 CPC 的離散表徵更擅長捕捉並對音位之間的關係分門別類。而此觀察也對應到這兩個模型較高的音位純度與相互資訊數值。

        接著考慮分群數的效應,我們進一步觀察分群表現最好的模型 HuBERT 在分群數為 50、100 和 200 的聯合分佈熱圖。圖 \ref{fig:hubert-comparison} 是三者的比較結果,從圖中可以發現,在分群數愈多時,熱圖較深的區域愈是可以集中連成一條線,落在線外的色塊變得更少,但每個格子的機率值也隨之迅速降低。這個趨勢可以解釋為什麼表 \ref{tab:single-cluster-results} 中上升的音位純度與下降的分群純度,不過從表格中可以發現,其實相互資訊的數值仍是隨著分群數上升而提高的,也就是分群數多時,可以幫助提升離散單元與音素標註之間的相關性。



{


\begin{table}[!htbp]
    \centering
    \begin{subtable}[t]{\textwidth}
        \centering
        \begin{tabular}{|c|c|c|c|c|c|} \hline
                        & 音位純度   & 分群純度   & 音位熵    & 離散單元熵  & PNMI   \\ \hline
            HuBERT      &     0.5256 &     0.3382 &    3.3152 &      3.8681 & 0.4993 \\ \hline    %% 1.6552 h
            CPC         &     0.5188 &     0.3812 &    3.3146 &      3.7918 & 0.4992 \\ \hline    %% 1.6545 c
            Wav2vec 2.0 &     0.4006 &     0.2676 &    3.3152 &      3.8215 & 0.3706 \\ \hline    %% 1.2286 w
            LogMel      &     0.3253 &     0.1473 &    3.3158 &      3.8630 & 0.2647 \\ \hline    %% 0.8776 l 
        \end{tabular}
        \caption{分群數 = 50}
        \label{tab:ch3-clu050-phn}
    \end{subtable}

    \vspace{0.5cm}

    \begin{subtable}[t]{\textwidth}
        \centering
        \begin{tabular}{|c|c|c|c|c|c|} \hline
                        & 音位純度   & 分群純度   & 音位熵    & 離散單元熵  & PNMI   \\ \hline
            HuBERT      &     0.6097 &     0.2553 &    3.3152 &      4.5704 & 0.5786 \\ \hline    %% 1.9181 h
            CPC         &     0.5895 &     0.2674 &    3.3146 &      4.5034 & 0.5557 \\ \hline    %% 1.8418 c
            Wav2vec 2.0 &     0.4877 &     0.2118 &    3.3152 &      4.5284 & 0.4596 \\ \hline    %% 1.5235 w
            LogMel      &     0.3348 &     0.0931 &    3.3158 &      4.5591 & 0.2789 \\ \hline    %% 0.9247 l 
        \end{tabular}
        \caption{分群數 = 100}
        \label{tab:ch3-clu100-phn}
    \end{subtable}

    \vspace{0.5cm}

    \begin{subtable}[t]{\textwidth}
        \centering
        \begin{tabular}{|c|c|c|c|c|c|} \hline
                        & 音位純度   & 分群純度   & 音位熵    & 離散單元熵  & PNMI   \\ \hline
            HuBERT      &     0.6474 &     0.1644 &    3.3152 &      5.2681 & 0.6289 \\ \hline    %% 2.0849 h
            CPC         &     0.6098 &     0.1789 &    3.3146 &      5.1885 & 0.5882 \\ \hline    %% 1.9497 c
            Wav2vec 2.0 &     0.5427 &     0.1467 &    3.3152 &      5.2173 & 0.5188 \\ \hline    %% 1.7199 w
            LogMel      &     0.3474 &     0.0569 &    3.3158 &      5.2322 & 0.2955 \\ \hline    %% 0.9798 l 
        \end{tabular}
        \caption{分群數 = 200}
        \label{tab:ch3-clu200-phn}
    \end{subtable}

    \caption{四種語音表徵在不同分群數的純度與相互資訊數據}
    \label{tab:single-cluster-results}
\end{table}

}  % table
{

% \newcommand{\jeffheightt}[1]{\includegraphics[width=0.6\linewidth]{#1}}
\newcommand{\jeffheightt}[1]{\includegraphics[width=1\linewidth]{#1}}

\begin{figure}
     \centering
     \begin{subfigure}{\textwidth}  % [t]{\textwidth}
         \centering
         \jeffheightt{figures/hubert-50-joint-byprob--new1.png}
         \caption{HuBERT}
         \label{fig:ch3-heatmap-model--hubert-50-joint-byprob}
     \end{subfigure}
     \vfill

     \begin{subfigure}{\textwidth}  % [t]{\textwidth}
         \centering
         \jeffheightt{figures/cpc-50-joint-byprob.png}
         \caption{CPC}
         \label{fig:ch3-heatmap-model--cpc-50-joint-byprob}
     \end{subfigure}

     % \vfill
     % \begin{subfigure}{\textwidth}  % [t]{\textwidth}
     %     \centering
     %     \jeffheightt{figures/cpc-50-joint-byprob.png}
     %     \caption{CPC}
     %     \label{fig:ch3-heatmap-model--cpc-50-joint-byprob}
     % \end{subfigure}
     % \vfill
     % \begin{subfigure}{\textwidth}  % [t]{\textwidth}
     %     \centering
     %     \jeffheightt{figures/logmel-50-joint-byprob.png}
     %     \caption{LogMel}
     %     \label{fig:ch3-heatmap-model--logmel-50-joint-byprob}
     % \end{subfigure}
     \caption{不同語音表徵在分群數為 50 的聯合分佈熱圖}
     \label{fig:ch3-heatmap-model-comparison--part1}
\end{figure}


\begin{figure}
    \ContinuedFloat
    % \setcounter{subfigure}{2}
     \centering
     % \begin{subfigure}{\textwidth}  % [t]{\textwidth}
     %     \centering
     %     \jeffheightt{figures/hubert-50-joint-byprob--new1.png}
     %     \caption{HuBERT}
     %     \label{fig:ch3-heatmap-model--hubert-50-joint-byprob}
     % \end{subfigure}
     % \vfill
     % \begin{subfigure}{\textwidth}  % [t]{\textwidth}
     %     \centering
     %     \jeffheightt{figures/w2v2-50-joint-byprob.png}
     %     \caption{Wav2vec 2.0}
     %     \label{fig:ch3-heatmap-model--w2v2-50-joint-byprob}
     % \end{subfigure}
     % \vfill
     
          \begin{subfigure}{\textwidth}  % [t]{\textwidth}
         \centering
         \jeffheightt{figures/w2v2-50-joint-byprob.png}
         \caption{Wav2vec 2.0}
         \label{fig:ch3-heatmap-model--w2v2-50-joint-byprob}
     \end{subfigure}
     
     \vfill
     \begin{subfigure}{\textwidth}  % [t]{\textwidth}
         \centering
         \jeffheightt{figures/logmel-50-joint-byprob.png}
         \caption{LogMel}
         \label{fig:ch3-heatmap-model--logmel-50-joint-byprob}
     \end{subfigure}
     \caption{不同語音表徵在分群數為 50 的聯合分佈熱圖(續)}
     \label{fig:ch3-heatmap-model-comparison--part2}
\end{figure}

}  % heatmaps

{

% \newcommand{\jeffheightt}[1]{\includegraphics[width=0.6\linewidth]{#1}}
\newcommand{\jeffheightt}[1]{\includegraphics[width=0.85\linewidth]{#1}}

\begin{figure}
     \centering
     \begin{subfigure}{\textwidth}  % [t]{\textwidth}
         \centering
         \jeffheightt{figures/hubert-50-joint-byprob--new2.png}
         \caption{分群數 = 50}
         \label{fig:ch3-heatmap-cluster--hubert-50-joint-byprob}
     \end{subfigure}
     \vfill

     \begin{subfigure}{\textwidth}  % [t]{\textwidth}
         \centering
         \jeffheightt{figures/hubert-100-joint-byprob---new2.png}
         \caption{分群數 = 100}
         \label{fig:ch3-heatmap-cluster--hubert-100-joint-byprob}
     \end{subfigure}

    \vfill

     \begin{subfigure}{\textwidth}  % [t]{\textwidth}
         \centering
         \jeffheightt{figures/hubert-200-joint-byprob.png}
         \caption{分群數 = 200}
         \label{fig:ch3-heatmap-cluster--hubert-200-joint-byprob}
     \end{subfigure}

     \caption{HuBERT 模型在不同分群數的聯合分佈熱圖}
     \label{fig:hubert-comparison}
\end{figure}

}
% 後面可以在有 unit 切入時,探討 given unit 時,為什麼分群數太多不好,而 100 又有什麼優於 50 的好處。(有必要再用 given phn 說明太分散)

        以上是不同離散表徵系統的離散單元對語音訊號給予虛擬標註時,對應音素標註是否明確的觀察。我們發現 HuBERT 是四種語音表徵之中效果最佳的,而分群數則是愈多愈好。

\subsection{以離散單元角度切入}  % 3 + 4 + 5  %%% 先看縱軸

  探討完綜觀機率分佈的比較,接著我們從離散單元的角度出發,基於離散單元進行統計觀察。首先,我們可以如同綜觀分析的探討方式,分別從模型與分群數量兩個變量切入,並對每個離散單元計算對應的條件音位熵 $H(y|z)$ 並畫出直方圖進行比較。

圖 \ref{fig:hist-model} 可以觀察到四種模型在分群數都是 50 時的條件音位熵直方圖,從圖中可以發現 HuBERT 和 CPC 的音位熵相較於 Wav2vec 2.0 和 LogMel 偏低,也就是 HuBERT 和 CPC 每個離散單元對應的音位相較集中,與綜觀探討得到的觀察相符。
{

% \newcommand{\jeffheightt}[1]{\includegraphics[width=0.6\linewidth]{#1}}
\newcommand{\jeffheightt}[1]{\includegraphics[width=0.51\linewidth]{#1}}

 % \newcommand{\jjvfill}{\vspace{0.05cm}}\renewcommand{\arraystretch}{0.7} % 調整行高
 \newcommand{\jjvfill}{\vfill}

\begin{figure}
     \centering
     \begin{subfigure}{\textwidth}  % [t]{\textwidth}
         \centering
         \jeffheightt{figures/histo-phngivenunitent-hubert50-cnt.png}
         \caption{HuBERT}
         \label{fig:ch3-heatmap-model--hubert-50-joint-byprob-hist}
     \end{subfigure}
     \jjvfill

     \begin{subfigure}{\textwidth}  % [t]{\textwidth}
         \centering
         \jeffheightt{figures/histo-phngivenunitent-cpc50-cnt.png}
         \caption{CPC}
         \label{fig:ch3-heatmap-model--cpc-50-joint-byprob-hist}
     \end{subfigure}

     % \caption{不同語音表徵在分群數為 50 的條件音位熵直方圖}
     % \label{fig:hist-model}
% \end{figure}


% \begin{figure}
    % \ContinuedFloat
     % \centering
     \jjvfill

          \begin{subfigure}{\textwidth}  % [t]{\textwidth}
         \centering
         \jeffheightt{figures/histo-phngivenunitent-w2v2_50-cnt.png}
         \caption{Wav2vec 2.0}
         \label{fig:ch3-heatmap-model--w2v2-50-joint-byprob-hist}
     \end{subfigure}
     
     \jjvfill
     \begin{subfigure}{\textwidth}  % [t]{\textwidth}
         \centering
         \jeffheightt{figures/histo-phngivenunitent-logmel50-cnt.png}
         \caption{LogMel}
         \label{fig:ch3-heatmap-model--logmel-50-joint-byprob-hist}
     \end{subfigure}
     \caption{不同語音表徵在分群數為 50 的條件音位熵直方圖
     % (續)
     }
     % \label{fig:hist-model--part2}
     \label{fig:hist-model}
\end{figure}

}  % heatmaps


%%%%%
圖 \ref{fig:hist-cluster} 則是比較 HuBERT 模型在分群數為 50、100 和 200 時的條件音位熵。需注意的是,由於此時離散單元數量不同,因此直方圖的縱軸改以比例數值呈現,亦即將數量分別除以 50、100 和 200 以進行公平的比較。從圖中可以觀察到,分群數愈多確實使整體條件音位熵降低,也與綜觀探討得到的小結一致。

{

% \newcommand{\jeffheightt}[1]{\includegraphics[width=0.6\linewidth]{#1}}
\newcommand{\jeffheightt}[1]{\includegraphics[width=0.75\linewidth]{#1}}

\begin{figure}
     \centering
     \begin{subfigure}{\textwidth}  % [t]{\textwidth}
         \centering
         \jeffheightt{figures/histo-phngivenunitent-hubert50.png}
         \caption{分群數 = 50}
         \label{fig:ch3-heatmap-cluster--hubert-50-joint-byprob-hist}
     \end{subfigure}
     \vfill

     \begin{subfigure}{\textwidth}  % [t]{\textwidth}
         \centering
         \jeffheightt{figures/histo-phngivenunitent-hubert100-prob.png}
         \caption{分群數 = 100}
         \label{fig:ch3-heatmap-cluster--hubert-100-joint-byprob-hist}
     \end{subfigure}

    \vfill

     \begin{subfigure}{\textwidth}  % [t]{\textwidth}
         \centering
         \jeffheightt{figures/histo-phngivenunitent-hubert200-prob.png}
         \caption{分群數 = 200}
         \label{fig:ch3-heatmap-cluster--hubert-200-joint-byprob-hist}
     \end{subfigure}

     \caption{HuBERT 模型在不同分群數的條件音位熵直方圖}
     \label{fig:hist-cluster}
\end{figure}

}

接著,由於本小節基於離散單元的角度,我們仿照前作如 SpeechTokenizer \cite{zhang2024speechtokenizer}、DinoSR \cite{liu2024dinosr} 的作法,將熱圖改以 $p_{y|z}(i|j)$ 呈現,即對每個直行進行標準化得到條件機率,以顯示每個單位對應到哪個音位,探討這種對應分佈是如何的集中或分散。

% /////////////
\begin{figure}
    \centering
    \includegraphics[width=1\linewidth]{figures/11111111.png}
    \caption{HuBERT 模型、分群數為 50 之 $p_{y|z}(i|j)$ 條件機率分佈圖}
    \label{fig:hubert-50-givenunit-byprob}
\end{figure}
        觀察圖 \ref{fig:hubert-50-givenunit-byprob} 中由左上而右下角對應的連線區域,首先我們會在左上方觀察到一條明顯較深的區域,也就是模型會安排一定數量的離散單元用以對應實際上並非音位的音素標註 sil。此外,我們還可以在連線區域之外觀察到一些零星的色塊,在此指示存在不少離散單元,它們對應的音位是相比較為分散的,也因此使得音位純度無法到達 100\%。
        
        不過,如果我們嘗試觀察這些將離散單元對應機率分散出去的音位,藉助語音學的知識可以觀察到一些有趣的發現:這些音位彼此之間在發音上具有很強的關聯性,幾乎與語音學提供的分類是對應的。 


        \begin{figure}
            \centering
            \includegraphics[width=1\linewidth]{figures/unit_rank_phn.png}  % figures/unit_perspective.png
            \caption[]{
 HuBERT 模型、分群數為 50 之部分離散單元所對應的前五高機率音位
% 。圖中的方框、圓圈等形狀
}
                                          % 表示輔音發聲部位,外框顏色則表示清濁音。注意元音都屬於濁音
                                          (音位分類以顏色標示區分)
            \label{fig:unit-to-phn-rankings}
        \end{figure}
        
\begin{figure}
    \centering
    \includegraphics[width=1\linewidth]{figures/ipa_similarity.png}
    \caption[]{
國際音標表的輔音表格,說明離散單元}
                                                                對語音聲學特徵的捕捉並不僅限單一面向
    \label{fig:ipa-cons-table-sim}
\end{figure}

        為了方便說明,我們將熱圖上各個離散單元排名前五高的對應音位另外列表呈現在圖 \ref{fig:unit-to-phn-rankings} 中,並用顏色標明各音位所屬的音位分類。從表中大致可以看出前幾高機率的音位所屬的類別確實是相近的。
而且即便不是同一個音位分類,
這些音位在語音學中,仍有其他
層面 --- 如發音部位和清濁音 --- 
的相似性,還是可以將各離散單元的前幾高音位中找出共通點。

事實上,為了作圖與統計方便,語音處理相關研究 \cite{10097097, abdullah23_interspeech} 
對音位的歸類是相對簡化的。根據
語音學的知識,
音位之間比分組方式不只有一種,而本研究著重的分類方式僅是以「發音方式」為主。
例如 05 號單元對應的前兩名 /t/ 和 /s/ 雖然並不屬於同一個發聲方式,因而被分成兩個類別,但如果參考圖 \ref{fig:ipa-cons-table-sim} 的國際音標表 \footnote{表中的每個橫列約等於本論文與相關研究\cite{10097097, abdullah23_interspeech} 使用的「發音方式」分類法,而每個用直線隔開的直行則是指「發音部位」相同,同一個格子的左右則是呈現一對清濁音音位。},
會發現它們都屬於「齒音」,亦即它們的「發音部位」是相同的。換言之這些離散單元捕捉到的語音特性是多個面向的,並不僅限於單一的分類方式,而是可以對應到國際音標表上至少兩個維度以上的類型。



        透過以上的觀察,因此我們有足夠的理由重新對熱圖的縱軸重新排列,並按照語音學分類進行分組,來觀察這些離散單元是如何指示出音位之間的相似性,區分出同個音位、同類發音,或者如何被混淆為其他類別。  %,而這些類別是否有某些特徵,最後這樣的現象是否只在單一模型出現,抑或是在不同的離散單元系統都會發生。

\begin{figure}
    \centering
    \includegraphics[width=1\linewidth]{figures/hubert-50-givenunit-byphn.png}
    \caption[]{% \medskip % \small
        HuBERT 模型、分群數為 50 之離散單元與音位標註的條件機率,}
                                                    依照韋氏(Wells) \cite{wells_phonetic_2022} 論文與音位類別排序的分佈圖
    \label{fig:hubert-50-givenunit-byphn}
\end{figure}

        圖 \ref{fig:hubert-50-givenunit-byphn} 的分組順序是依照韋氏(Wells) \cite{wells_phonetic_2022} 論文中的出現順序排列,而組別內則是清音在上、濁音在下,而同樣清濁音則是以發音位置由前往後排列。除了縱軸上按照音位本身特性分組,依循純度中使用的「代表音位」 $i^\ast$ 概念,我們同樣也對每個離散單元的代表音位排序,並且也依照這些代表音位進行分組觀察。

最後,對於每個離散單元 $j$ 與對應的最高機率音位 $y^*(j)$,為了統計該單元 $j$ 除了 $y^*(j)$ 以外,是否也給予與 $y^*(j)$ 同音位分類 $\kappa^*(j)$ 的其他音位較高的機率值,我們藉由調整純度的計算式,但將音位標註改為音位分類並重新統計,以方便我們比較離散單元「對音位分類歸類能力」的強弱。計算方式為:

如果 $y^*(j) \in \kappa^*(j)$,$ \kappa^*(j) = \{y^*(j), y', y'', \cdots\cdots\}$  是所有與  $y^*(j)$ 同音位分類的音位,則將這些音位的標註改為 $ \kappa^*(j) $ ,統計音位分類純度
\begin{align}
    \mathbb{E}_{p_z(j)}\left[p_{ \kappa|z}( \kappa^*(j)|j) \right]
\end{align}
與音位分類的分群純度
\begin{align}
    \mathbb{E}_{p_ \kappa(\kappa^*)}\left[p_{z| \kappa}(z^*( \kappa^*)|\kappa^*) \right]
\end{align}
以此刻劃離散表徵是否能歸類語音學上相似的發音特徵。


\begin{figure}
        \centering
        \includegraphics[width=0.55\linewidth]{figures/pcls-pur.png}
    \caption{以音位分類為標註計算,四種語音表徵在不同分群數的純度數據}
    \label{fig:pcls}
    \end{figure}

圖 \ref{fig:pcls} 是不同離散表徵在音位分類純度與對應的分群純度結果。由表中可以再次確認,HuBERT 的離散單元不但能夠很好的區分出音位,即便某些離散單元沒有集中分到特定音位之上,也可以很不錯的給予同類別的音位較高的機率值,以得到較高的音位分類純度數值。

% sil 沒有必要寫了!(說不定 purity cls 也可以拿掉)


\subsection{以音位角度切入}

  接著,我們改從音位的角度切入,觀察每個音位所對應的離散單元條件熵 $H(z|y)$ ,觀察不同音位之間是否有特定幾個音位較容易或難以被離散表徵歸類。表 \ref{fig:phn-specials} 分別呈現不同模型在分群數為 50 和 100 時,離散單元熵最高與最低的幾個音位。雖然沒有特別明顯,但由此大致可以看到:

\begin{itemize}
    \item 熵值較低的音位約有 AA、EY、F、ZH、SH、S 等,其中 F、ZH、SH、S 皆屬擦音
    \item 熵值較高的音位則有 spn、AH、IH、T、D 等,其中 T、D 等屬於塞音  % 機率高低先跳過了
\end{itemize}

整體而言,擦音的離散單元相對較為集中,而塞音則相對較為分散。而且這個趨勢在不同的語音表徵和分群數都約略存在,只是以 HuBERT、CPC 較為明顯。


{

% \newcommand{\jeffheightt}[1]{\includegraphics[width=0.6\linewidth]{#1}}
\newcommand{\jeffheightt}[1]{\includegraphics[width=1\linewidth]{#1}}

\begin{figure}
     \centering
     \begin{subfigure}{\textwidth}  % [t]{\textwidth}
         \centering
         \jeffheightt{figures/phnrank50.png}
         \caption{分群數 = 50}
         \label{fig:phn-specials-clu50}
     \end{subfigure}
     \vfill

     \begin{subfigure}{\textwidth}  % [t]{\textwidth}
         \centering
         \jeffheightt{figures/phnrank100.png}
         \caption{分群數 = 100}
         \label{fig:phn-specials-clu100}
     \end{subfigure}

     \caption{不同語音表徵之離散單元熵最高與最低的幾個音位}
     \label{fig:phn-specials}
\end{figure}

}



\begin{figure}
    \centering
    \includegraphics[width=1\linewidth]{figures/better-demo-splitter.png}
    \caption{對聯合機率分佈按照音位分類分別計算純度作法示意圖}
    (以塞音舉例,同樣的作法對紅線分開的八塊區域分別計算)
    \label{fig:demo-splitter}
\end{figure}


為了進一步驗證不同音位分類之間的差異,我們可以再度引用純度與相互資訊的計算公式,但此時將統計的範圍限定為不同的音位分類,分別計算出針對每一個音位分類的純度與相互資訊。換言之,我們可以將機率的聯合分佈圖 $p_{yz}$ 按照圖 \ref{fig:demo-splitter} 的紅色水平線分成八塊後,重新標準化並各自計算純度與相互資訊,等同於將原本的語音音框按照音位分類分成八組各自統計這些指標。如此一來,一方面可以依據每個音位分類,各自觀察在不同離散表徵對待該類別表現的差異;另一方面,也可以藉由比較不同音位分類彼此的整體趨勢,歸納音位分類本身發音特徵被捕捉的難易程度。

\begin{figure}
    \centering
    \includegraphics[width=1\linewidth]{figures/pur-of-each-cls-xls.png}
     \caption{按照音位分類分開各自計算的純度與相互資訊}
     \label{fig:pur-of-each-cls}
\end{figure}


        表 \ref{fig:pur-of-each-cls} 中便呈現了這些模型的比較數據。由圖中依然可以觀察到 HuBERT 優於其他模型,且分群數愈多時相互資訊與音位純度愈高,這些趨勢依然與前面所有的觀察得到的結論一致。

而從音位分類之間的比較,我們還可以觀察到:撇去非音位(表格中的「XXX」)的類別由於只有 sil 一類標註,因此相互資訊和純度相當高之外,
我們可以確認:塞音和塞擦音的純度較低,確實是比較難以集中歸類的音位分類;而近音、雙元音和擦音則純度相對較高,也驗證為什麼它們的離散單元熵值較低、分佈較為集中。

\subsection{整體熱圖驗證}


{

% \newcommand{\jeffheightt}[1]{\includegraphics[width=0.6\linewidth]{#1}}
\newcommand{\jeffheightt}[1]{\includegraphics[width=0.75\linewidth]{#1}}

\begin{figure}
     \centering
     \begin{subfigure}{\textwidth}  % [t]{\textwidth}
         \centering
         \jeffheightt{figures/badhub50.png}
         \caption{HuBERT + 分群數 = 50}
         \label{fig:ch3-badhub50}
     \end{subfigure}
     \vfill

     \begin{subfigure}{\textwidth}  % [t]{\textwidth}
         \centering
         \jeffheightt{figures/badhub100.png}
         \caption{HuBERT + 分群數 = 100}
         \label{fig:ch3-badhub100}
     \end{subfigure}

    \vfill

     \begin{subfigure}{\textwidth}  % [t]{\textwidth}
         \centering
         \jeffheightt{figures/badcpc100.png}
         \caption{CPC + 分群數 = 100}
         \label{fig:ch3-badcpc100}
     \end{subfigure}

     \caption{熱圖驗證塞音、塞擦音較難以被離散單元歸類,}
     注意塞擦音在 HuBERT + 分群數 = 50 和 CPC + 分群數 = 100 \\
     甚至可能沒有專門的離散單元以其為代表
     \label{fig:finalObserv}
\end{figure}

}


{

% \newcommand{\jeffheightt}[1]{\includegraphics[width=0.6\linewidth]{#1}}
\newcommand{\jeffheightt}[1]{\includegraphics[width=0.75\linewidth]{#1}}

\begin{figure}
     \centering
     \begin{subfigure}{\textwidth}  % [t]{\textwidth}
         \centering
         \jeffheightt{figures/goodhub50 - 複製.png}
         \caption{HuBERT + 分群數 = 50}
         \label{fig:ch3-goodhub50}
     \end{subfigure}
     \vfill

     \begin{subfigure}{\textwidth}  % [t]{\textwidth}
         \centering
         \jeffheightt{figures/goodhub100 - 複製.png}
         \caption{HuBERT + 分群數 = 100}
         \label{fig:ch3-goodhub10}
     \end{subfigure}

    \vfill

     \begin{subfigure}{\textwidth}  % [t]{\textwidth}
         \centering
         \jeffheightt{figures/goodcpc100 - 複製.png}
         \caption{CPC + 分群數 = 100}
         \label{fig:ch3-goodcpc10}
     \end{subfigure}

     \caption{熱圖驗證擦音、雙元音與近音的特徵較明顯}
          \label{fig:finalObserv2}
\end{figure}

}


  最後,參考韋氏(Well) \cite{wells_phonetic_2022} 的研究方法,我們可以探討不同音位分類中音位與離散單元的對應關係。然而比起直接看離散單元的編號,在此我們改由對機率熱圖做分區觀察以確認趨勢。在此,為了可以同時對比語音表徵與分群數兩個變因造成的差異,我們比較 HuBERT分群數 50 和 100 以及 CPC分群數 100 的機率熱圖,並參考 SpeechTokenizer \cite{zhang2024speechtokenizer}、DinoSR \cite{liu2024dinosr} 的作法以 $p_{y|z}(i|j)$ 呈現,確認離散表徵對於音位的歸類效果,最後確認前面的所有觀察。圖 \ref{fig:finalObserv} 所框出的區域為前面觀察到較為分散的塞音與塞擦音,確實為顏色較淺的區域(塞擦音在 HuBERT + 分群數 = 50 和 CPC + 分群數 = 100 甚至可能沒有專門的離散單元以其為代表),證明其語音特徵歸類的確較為困難;而圖 \ref{fig:finalObserv2} 所框出的區域則是離散表徵歸類比較集中的擦音、雙元音與近音,這幾區的色塊也如前面推論所預測的較為明顯,屬於容易區分出來的音位分類。



% 最後,關於非音位的語音訊號,觀察這些模型的等效花費多少比例的離散單元去代表這些非音位的資訊,可以發現 HuBERT 雖然擷取音位的整體表現比較好,但也耗費了不低比例的單元去編碼非音位的資訊。


{

\section{本章總結}


  本章節探討以音框為單位取出的語音離散表徵與對應的音位標註之間的關係。我們從純度的計算開始,對於整個機率熱圖做了視覺化分析,並且透過語音知識的協助,尤其是對於音位的分類,將原先約 40 類的獨立標註,進行更深入的特性分析。

藉由這些探討,一方面我們得出如塞音的離散單元分佈較為分散、擦音比較集中等針對音位特性的觀察,同時也確認了 HuBERT 模型的離散表徵,在各項數據中展現與音位之間相似性最為明顯。由此,我們進一步印證為何 HuBERT 是抽取語音離散表徵時最常使用的模型,並常被無文字架構所使用。

        然而,單一離散表徵僅能代表 10 或 20 毫秒的語音訊號,而音位的長度經常佔據不只一個離散表徵。因此,下一章節將嘗試進一步組合多個離散表徵成為符記,分析它們與音位之間的關係。

%   從統計數據出發,我們針對聯合分佈的各個面向,配合了語音學知識的分類進行了細部探討,發現了 hunert 模型怎樣怎樣,而這件事可能在其他的模型之中差不多是 holid 住的。然後,因為 hunert 本身捕捉的各項純度與 MI 明顯較高,以此可以驗證為什麼 HuBERT 的離散單元可以在無文字架構內被當成類似音位或文字的表徵,並進而套用於語音語言模型的訓練上,同時為許多做語音模型解釋性的作品所關注citehao。


}
