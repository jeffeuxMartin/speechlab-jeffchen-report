\chapter{單一語音離散表徵與音位的關係}

  HuBERT 的成功,促使了語音離散表徵相關的研究。

\section{相關研究}

  相關的研究 \cite{10097097} 有從無文字的技術角度切入,以及解釋性和語音學的角度接入。

\section{衡量指標}

  我們在這邊探討的指標主要分為純度(Purity)和熵(Entropy)兩大類,以刻劃標註與離散單元之間的相關性。首先我們先說明音框指標是怎麼得來的,然後介紹了純度和熵等資訊理論相關的定義。

\section{語音學的音位分類(Phoneme Type)}

  為了描述清楚後續語音學知識相關的事情,在這邊先簡單介紹一下語音學上對音位是怎麼分類的。
%%%%%%%%%%%%%%%%%%%%%%%%%%%%%%%%%%%%%%%%%%%%%%%%%%%%%%%%%
        然而,除了這些分類本身,對於音位之間還有其他語音上的相似性,例如發音部位、清濁音等等特徵。為了方便分析,本次研究先以這些分類為主,而其他發音上的特徵則用在細部的個案探討上,以對整體解釋性有更全面的刻劃。

\section{實驗集與分析模型}

  本研究的分析對象參考無文字架構 \cite{noauthor_textless_2021, lakhotia_generative_2021, lakhotia_generative_2021-1} 的研究,分析了四種模型,也稍微介紹了一下。

\section{分析方式}

  針對模型得出之離散單元與音位標註之間的對應關係,為了更直觀的解釋這些指標的意義,並且看清楚這些數字背後之間代表的現象與細部特徵,我們將音位與離散單元的共同機率分佈 \(p_{yz}\) 用熱圖(Heatmap)呈現,來解釋這些指標的意義,並以此進一步往下深入探討。

        首先,這裡以 HuBERT 為基石模型、離散單元分群數為 50 的統計數據為例,圖 \ref{fig:hubert-50-joint-byprob} 說明我們如何分析語音離散表徵與音位標註的關係。

\begin{figure}
    \centering
    \includegraphics[width=1\linewidth]{figures/hubert-50-joint-byprob.png}
    \caption{HuBERT 模型、分群數為 50 之 \\
    離散單元與音位標註的共同機率分佈圖}
    \label{fig:hubert-50-joint-byprob}
\end{figure}

        圖中的縱軸表示各個音位,橫軸表示各個離散單元。在這張圖中,縱軸的音位是按照其邊際機率 \(p_y(i)\) 由高至低排序;橫軸的離散單元則是依據其對應的最高機率音位 \(y^\ast(j)\) 的縱軸排序位置進行排列。\footnote{如果兩個離散單元 \(j_1\) 和 \(j_2\) 對應到相同的音位 \(y^\ast = y^\ast(j_1) = y^\ast(j_2)\),則依照機率值 \(p_{yz}(y^\ast, j_1)\) 和 \(p_{yz}(y^\ast, j_2)\) 由高到低進行排序,對於多個離散單元的情況以此類推。} 這樣可以在熱圖上呈現出由左上至右下的對應關係。

        為了評估離散表徵是否有捕捉到與音位相關的資訊,我們可以分別從音位與離散單元的兩個角度出發,考慮以下兩個問題:
\begin{enumerate}
    \item 對於每個音位而言,它們所對應的離散單元集中程度如何?
從這個角度出發,可以觀察不同音位的集中程度,進而推論模型是否能夠辨認出該音位的發音特性,藉由給予夠高的一致性將這些語音訊號分類在一起。會不會有某一些音位很難被歸類出來?
    \item 反之,一個離散單元所對應的音位的集中或分散程度如何?
如果一個音框的語音訊號被模型指示為特定的離散單元,該單元作為虛擬標註,能多大程度的對應到人耳感知的音位標註?也就是這些虛擬標註,是否達成音位標註類似的效果,足以把不同語音特徵區分開來。
\end{enumerate}

        這兩個問題的答案可以分別很直觀的從圖中顏色的深淺觀察出來,也正好對應前面所提及的兩個純度指標:
        \begin{enumerate}
            \item 將每個橫列(Row)取最大值相加後的總和即為分群純度
            \item 將每個直行(Column)取最大值相加後的總和則是音位純度
        \end{enumerate}

        從這裡我們可以看到,當分群數量增加時,音位純度可以在每個直行上取到更多的機率值,這也意味著當分群數量與音框數量相同時,音位純度可以達到 100\%,與前面的描述互相吻合。另一方面,當分群數量增加時,每個格子的機率值會因為離散單位數量的增多而被稀釋,而分群純度受到音位數量限制,只能取 41 個 $p_{yz}$ 值的總和,使得單位純度因而明顯降低。

        以上是綜觀整個系統給予虛擬標註時,對應到音位標註的好壞。然而我們可以更進一步的探討各音位與離散單元之間的內部差異,也就是分別探討:
\begin{itemize}
    \item 哪些離散單元比較能集中抓取音位的特徵,不會與其他音位混淆?
    \item 如果一個離散單元被分散的對應到多個音位,那麼這些音位可能是哪幾個?是否存在某些共同特徵?
\end{itemize}

        由於這些問題是以離散單元的角度出發,因此我們仿照前作如 SpeechTokenizer \cite{zhang2024speechtokenizer}、DinoSR \cite{liu2024dinosr} 的作法,將熱圖改以 $p_{y|z}(i|j)$ 呈現,即對每個直行進行標準化得到條件機率,以顯示每個單位對應到哪個音位,探討這種對應如何集中或分散。

\begin{figure}
    \centering
    \includegraphics[width=1\linewidth]{figures/11111111.png}
    \caption{HuBERT 模型、分群數為 50 之\\
$p_{y|z}(i|j)$  條件機率分佈圖}
    \label{fig:hubert-50-givenunit-byprob}
\end{figure}
        從圖 \ref{fig:hubert-50-givenunit-byprob} 中可以更明顯的看出,模型會耗費不少種類的離散單元於編碼非音位的音素標註(尤其是 sil)之上。
% 加上 silence ratio?
此外,每個離散單元對於其對應的訊號所對應的音位集中程度有高有低,使得音位純度無法到達 1.00。然而,這邊比較有趣的點是,觀察那些對應音位比較分散的離散單元,我們其實可以發現這些音位彼此之間有很強的關聯性,幾乎與前述的語音分類一致。
        % \begin{figure}
        %     \centering
        %     \includegraphics[width=1\linewidth]{figures/unit_perspective.png}
        %     \caption{離散單元對應的前幾高音位示意。圖中的方框、圓圈等形狀表示輔音發聲部位,外框顏色則表示清濁音。注意元音都屬於濁音}
        %     \label{fig:unit-to-phn-rankings}
        % \end{figure}

\begin{figure}
    \centering
    \includegraphics[width=1\linewidth]{figures/unit_rank_phn.png}
    \caption[]{
    離散單元對應的前幾高音位示意。圖中的方框、圓圈等形狀}
          表示輔音發聲部位,外框顏色則表示清濁音。注意元音都屬於濁音
    \label{fig:unit-to-phn-rankings}
\end{figure}
        
        這件事可以從熱圖上由左上而右下連線中,不在線上但顏色較深的區塊中觀察出來。但由於直接從熱圖上觀察比較難以呈現,因此我們另外統計出表 \ref{fig:unit-to-phn-rankings},其中展現的是幾個離散單元對應的前五高機率音位,並且用顏色標明各音位所屬的語音學類別。從表中大致可以看出以上描述的趨勢,而且即便不是同一個語音學類別,按照前面講解語音學對音位歸類的另外兩個層面 --- 發音部位和清濁音,還是可以將各離散單元的前幾名之中盡量找出共通點。例如 05 號單元對應的前兩名 /t/ 和 /s/ 雖然並不屬於同一個發聲方式,因而被分成兩個類別,但如果從國際音標表中的「發音部位」來觀察,會發現它們都屬於「齒音」。換言之這些離散單元捕捉到的語音特性是多個面向的,並不僅限於單一的分類方式,而是可以對應到國際音標表上至少兩個維度以上的類型。

\begin{figure}
    \centering
    \includegraphics[width=1\linewidth]{figures/ipa_similarity.png}
    \caption[]{
    國際音標表的輔音表格,說明離散單元}
                                對語音聲學特徵的捕捉並不僅限單一面向
    \label{fig:ipa-cons-table-sim}
\end{figure}

        透過以上的觀察,因此我們有足夠的理由重新對熱圖的縱軸重新排列,並按照語音學分類進行分組,來觀察這些離散單元是如何指示出音位之間的相似性,區分出同個音位、同類發音,或者如何被混淆為其他類別,而這些類別是否有某些特徵,最後這樣的現象是否只在單一模型出現,抑或是在不同的離散單元系統都會發生。

\begin{figure}
    \centering
    \includegraphics[width=1\linewidth]{figures/hubert-50-givenunit-byphn.png}
    \caption[]{
        HuBERT 模型、分群數為 50 之離散單元
        }  % \medskip % \small
                       與音位標註的條件機率,依照語音學分類排序的分佈圖
    \label{fig:hubert-50-givenunit-byphn}
\end{figure}


        這張圖的分組順序是依照韋氏(Wells) \cite{wells_phonetic_2022} 論文中的出現順序排列,而組別內則是清音在上、濁音在下,而同樣清濁音則是以發音位置由前往後排列。除了縱軸上按照音位本身特性分組,依循純度中使用的「代表音位」 $i^\ast$ 概念,我們同樣也對每個離散單元的代表音位排序,並且也依照這些代表音位進行分組觀察。

        為了比較好的刻劃這個在分群內的好壞,我們接下來多算兩個指標:
\begin{enumerate}
    \item 語音分類的純度:為了確認每個離散單元如何「將音位至少分到同一語音學類別」的程度,藉由將前面音位純度的式子,但將音位標註改為語音學類別,便可以求得這個數據。
    \item 各發音類別的純度:為了衡量模型對於每個類別內部區分不同音位的能力,比較模型對於不同組別區分音位的難易度,我們可以根據音位的語音學類別,將所有的音框等效分成八份語料後,分別再次統計純度(亦即計算對語音學類別取條件機率後計算純度)。
\end{enumerate}

        說明完以上指標後,我們將展示不同離散表徵模型的所有分析,彼此先進行綜觀比較,此後再針對細部的特徵分析。

\section{分析結果}

\subsection{不同語音離散表徵的比較}

  首先是比較不同模型的離散特徵之數據與機率分佈圖:

純度:

\begin{figure}
    \centering
    \includegraphics[width=1\linewidth]{figures/000.png}
    \caption{比較表}
    \label{fig:enter-label}
\end{figure}

機率分佈圖:

\begin{figure}
    \centering
    \includegraphics[width=0.5\linewidth]{figures/hubert50.png}
    \caption{Hubert}
    \label{fig:enter-label}
\end{figure}
\begin{figure}
    \centering
    \includegraphics[width=0.5\linewidth]{figures/w2v250.png}
    \caption{w2v2}
    \label{fig:enter-label}
\end{figure}

\begin{figure}
    \centering
    \includegraphics[width=0.5\linewidth]{figures/cpc50.png}
    \caption{cpc}
    \label{fig:enter-label}
\end{figure}

\begin{figure}
    \centering
    \includegraphics[width=0.5\linewidth]{figures/logmel50.png}
    \caption{logmel}
    \label{fig:enter-label}
\end{figure}

        比較不同的模型的聯合分佈後,我們可以觀察到這些模型之間,確實存在。

        就是說標注跟單元之間的 correlation 高低,比較,從聯合分佈圖上也可以明顯的呈現出來,進而說明這些不同的語音離散表徵之前,捕捉訊號中的發音特徵能力的強弱。從圖中可以看出,Hubert 跟 CPC 的效果比另外兩種表徵好上不少。

        此外,我們也可以比較同一種語音表徵之下,不同的離散單元分群數之間對於音位特徵捕捉的強弱程度。基於前面表徵純度與相互資訊的考量,這邊固定用 hubert 當比較對象。從這三種分群數看來,離散單元數量愈多,愈能夠區分出比較細節的語音類別。例如,如果要從每個 unit 的代表音位來觀察,要至少有 100 個群數,才至少有一個離散單元能代表塞擦音。如果分群數量太少,很多細節的發音音位則很容易被放在一起,難以區別出更細節的發音差異。

        再者,我們觀察到模型會消耗一定比例的離散單元去代表非音位的發音,以此我們計算出一個 sil ratio,也就是等效有幾個 units 被拿去代表這些不是音位的聲音。具體算法是使用 E ( u goven p) 總和除以離散單元的數量。

        然後,我們期望看到一個單元系統怎麼有效的去使用這些離散單元,因此我們可以畫一張 histogram of entropy of units 來刻劃模型系統在使用這些離散單元的集中程度。如果整體向熵值低處偏,表示模型的各種離散單元所代表的音位都是相對確定的,也就是更好的捕捉到了音位特徵。

        最後,為了觀察各自不同語音學類別內分群的好壞,我們可以算語音學類別的純度,並且再以各個語音學類別區分出來,計算八個類別的各自 condition al 純度,以此比較不同表徵、分群數在針對不同語音學類別的特徵補綴效果。

        比較完這些表徵與離散單元數量的各項綜觀統計指標後,我們基於 MI 和 pur 的高,著重關注於 hubert 100 和 50 模型或 CPC,藉由觀察和比較兩個模型共有的規律,往下細部探討該模型所捕捉到的離散單元和因為之間的關係。

\subsection{個案探討}  % 分組討論

  首先,承接前面所述的「熱塗上不再斜線上的點」,我們可以觀察每個離散單元所對應的 phn 之間的共同特性。

        接著,我們確認一下對應最紛亂與最集中的幾個 unit 的狀況。

        接著,follow haotang 之前的作法,我們觀察各類別的 phn 可能各自是以那些離散單元為代表。

        之後,從各個 phn 的熵值來觀察,我們可以發現某某某某某幾個音位的熵值特高特地,來發現這些音位可能是比較難以捕捉的。而這個現象在 50 模型又可能是怎樣怎樣的。

\subsubsection{切塊出來}

  最後,基於我們已經有對語音學分類,我們可以觀察熱塗上不同語音學類別所切出來的區域的亂度,來觀察各類別的發音特徵捕捉的難易度。例如塞音怎樣怎樣啊……

\section{結論}

  從統計數據出發,我們針對聯合分佈的各個面向,配合了語音學知識的分類進行了細部探討,發現了 hunert 模型怎樣怎樣,而這件事可能在其他的模型之中差不多是 holid 住的。然後,因為 hunert 本身捕捉的各項純度與 MI 明顯較高,以此可以驗證為什麼 HuBERT 的離散單元可以在無文字架構內被當成類似音位或文字的表徵,並進而套用於語音語言模型的訓練上,同時為許多做語音模型解釋性的作品所關注citehao。

\newpage\textcolor{white}{0}\newpage  %

\mychcnt{3}
\chapter{單一語音離散表徵與音位的關係}

\input{chapters/do_ch3}
