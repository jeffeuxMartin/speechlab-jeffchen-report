
{

        % 對於一個聲學單元發掘的系統,我們關注的問題會在


    對於一個 unit encoding 系統的評估,我們特別關注的是每個 unit 所代表的 phoneme 是否集中,以及每個 phoneme 是否可以由較少的 unit 代表。理想情況下,當某個 unit 出現時,我們應該幾乎能夠確定它代表的是哪一個 phoneme。

這兩個指標可以在圖上通過顏色深淺來直觀地展示:
首先,phoneme purity 表示每個 column 取最大值後相加的總和。其次,unit purity 表示每個 row 取最大值後的 pxy 總和。 

從這裡我們可以看到,unit purity 受到每個 unit 只能取 41 個 pxy 總和的限制。而當 cluster 數量增加時,phoneme purity 可以取得更多的 pxy 總和。這也意味著當 cluster 數量與 frame 數量相同時,phoneme purity 可以達到 100\%。另一方面,當 cluster 數量增加時,每個格子的 pxy 會因 unit 數量的增多而被稀釋,但 unit purity 的 pxy 總和受到 phoneme 數量的限制,因此當 cluster 數量增加時,unit purity 會降低,這一點可以從圖中觀察到。

參考浩然哥和 sptok 等前人的研究,由於 unit 是假設,我們主要關注的是每個 unit 如何對應到音位標註上。為了更直觀地觀察這一點,我們將熱圖呈現為 p(p|u),即對每個值進行歸一化,以顯示每個 unit 對應到哪個 phoneme,以及這種對應是集中還是分散。如下圖 \ref{fig:prob} 所示:

}

        如何
zz