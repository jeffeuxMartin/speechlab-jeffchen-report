% \chapter{多個語音離散表徵與音位的關係}

\newcommand{\jefftablesep}{\vspace{0.5cm}}\renewcommand{\arraystretch}{0.7} % 調整行高

\section{動機}

  如前一章所述,一個文字或音位往往對應到上百毫秒的語音訊號,然而單一離散單元所對應的聲音訊號為 10 或 20 毫秒,亦即同一段語音所對應的離散單元數目將比音位或文字多出許多。本章節從自然語言處理中獲取靈感,將\textcolor{red}{分詞演算法(Tokenization)\textcolor{red}{(tokenization 的翻譯需要調整!)}}應用於離散單元序列上,並應用上一章節的分析方法檢驗將多個離散單元所組成之符記。探討分詞後,離散單元是否可以\textcolor{red}{同時擁有無文字(Textless)}\cite{lakhotia_generative_2021, lakhotia_generative_2021-1, noauthor_textless_2021} 的特性,且更接近音位的序列,成為更好的語音表徵。

\section{相關研究} 

  在無文字架構被提出後的約兩年後,藉分詞方法組合離散單元的研究逐步出現。最初提出「聲學片段(Acoustic Piece)」的是任氏(Ren)等人 \cite{ren_speech_2022}\citetag{1-22A4-Pretrain-ap}),該論文比對離散單元序列及對應的文字轉寫,從中觀察到許多相似的規律重複出現,而且不限於單一語者。受此啟發,本論文首先將離散單元使用句子片段(SentencePiece) \cite{kudo_sentencepiece_2018} 分詞,獲得新的符記 --- 「聲學片段」,並用於語音辨識的預訓練上。

        不久,由吳氏(Wu)提出的 Wav2seq \cite{wu_wav2seq_2023}\citetag{2-22A5-wav2seq}論文中,考量文字與語音的序列長度差異,並基於離散單元和音位的關聯性,將離散單元視為字符(Character),嘗試將這些字符透過分詞方法組成「虛擬語言(Pseudo-language\footnote{偽語言對應之離散單元被視為「虛擬文字(Pseudo-text)」})」,來幫助語音到文字的模型。因為解碼器在實際應用時需要生成的序列多是文字的符記 --- 次詞單位(Subword Unit),因此該篇研究旨在讓模型在預訓練後可以快速適應下游任務。與前一篇呼應,「聲學片段」對語音預訓練的效果在\cite{10096788}\citetag{3-23A-coarser-grain}中被探討,此後聲學片段更被應用於縮短資料序列長度\cite{chang_exploration_2023}\citetag{4-23B-Exploration of Efficient End-to-End ASR using Discretized Input from Self-Supervised Learning} 、語音生成\cite{shen2024acoustic}\citetag{5-24A-speech-gen},或學習更穩健(Robust)的語音表徵\cite{chang2023r}\citetag{6-23B-rspin-acousticpiece}。

        近期,張氏(Chang)等人\cite{chang_exploring_2024}\citetag{7-23B-shinji-hsiuhsuan}將以分詞方法處理離散單元的流程(Pipeline)納入 ESPNet 套件 \cite{watanabe2018espnet} 中,並在語音辨識、語音翻譯等任務中獲得了超越以往的表現,進一步證明了這個方法的效果。

\section{分詞方法}

  在以文字為主體的自然語言處理中,文字文本除了以單詞(Word)或字元(Character)為處理單位,更常見的作法是透過分詞演算法(Tokenization)將文本分段,以「次詞單位」構成詞彙表來重新編碼文本,用於文字模型的訓練與推理。

        分詞方法的優點一般包含:

\begin{enumerate}
    \item 固定詞彙表大小,避免未登錄詞(Out-of-vocabulary,OOV)
    \item 縮短資料序列的長度,提升訓練和推論的效率。
    \item 分解單詞,捕捉更細緻的語意關係,模擬如英語中的字首(Prefix)、字尾(Suffix)等等具有特定意義的文字組合。
\end{enumerate}

\subsection{常見演算法}

  以下介紹幾種常見的分詞方法:

\paragraph{位元組對編碼(Byte Pair Encoding,BPE)} \hfill \break
%
  位元組對編碼 \cite{10.5555/177910.177914, sennrich_neural_2016} 是一種常用的分詞方法,最初來自資料壓縮技術 \cite{10.5555/177910.177914},後來被引入到自然語言處理領域,用以處理機器翻譯問題 \cite{sennrich_neural_2016} 。
該演算法從字元開始,根據詞彙表中各個次詞單位的頻率,反覆合併常見的字元成為新的次詞單位,直到達到預定的詞彙表大小。

\paragraph{單詞片段(WordPiece)} \hfill \break
%
  WordPiece \cite{wu2016google} 演算法由 Google 用以訓練機器翻譯系統,並在 BERT \cite{devlin_bert_2019} 模型中被使用而廣為人知。與位元組對編碼相似,同樣是透過反覆合併的策略,但合併的依據改以機率模型取代出現頻率。

\paragraph{單一詞語言模型(Unigram Language Model)} \hfill \break
%
  單一詞語言模型 \cite{kudo2018subword} 是基於語言模型的分詞方法,以機率分佈選擇次詞單位,並以最大化輸入文本的機率來為文本分段。

\subsection{句子片段(SentencePiece)套件}

  SentencePiece \cite{kudo_sentencepiece_2018}
是由 Google 開發的分詞套件,實作了前述的位元組對編碼和單一詞演算法。其優勢在於可應用於不同語言,尤其用於處理中文、日文等不使用空格分隔單詞的語言文本時,此套件大大的簡化了前處理的流程。

\section{衡量方式}

  本章節沿用上一章節 LibriSpeech 資料集的 train-clean-100 訓練子集,以及相同的分析數據以進行比對。由於與上一章節的差異僅在分詞方法的引入,因此數據結果將多紀錄分詞前後序列長度的變化;其他操控變因包含分詞方法與詞表大小。考慮到語音訊號本身不如英語等文字,在書寫時就已經具備空格分隔單詞,因此以下分析結果皆採用 SentencePiece 套件中實作之單一詞演算法為分詞方法,並比較詞表大小 500、1000、8000、10000、20000 五種設定的結果差異。(空間不足時則僅呈現 500、1000、10000 三種設定的趨勢。)
