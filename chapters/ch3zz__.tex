\chapter{單一語音離散表徵與音位的關係} \input{chapters/czch3}

{

\section{分析方式}

  針對模型得出之離散單元與音位標註之間的對應關係,為了更直觀的解釋這些指標的意義,並且看清楚這些數字背後之間代表的現象與細部特徵,我們將音位與離散單元的共同機率分佈 \(p_{yz}\) 用熱圖(Heatmap)呈現,來解釋這些指標的意義,並以此進一步往下深入探討。

        首先,這裡以 HuBERT 為基石模型、離散單元分群數為 50 的統計數據為例,圖 \ref{fig:hubert-50-joint-byprob} 說明我們如何分析語音離散表徵與音位標註的關係。

\begin{figure}
    \centering
    \includegraphics[width=1\linewidth]{figures/hubert-50-joint-byprob.png}
    \caption{HuBERT 模型、分群數為 50 之 \\
    離散單元與音位標註的共同機率分佈圖}
    \label{fig:hubert-50-joint-byprob}
\end{figure}

        圖中的縱軸表示各個音位,橫軸表示各個離散單元。在這張圖中,縱軸的音位是按照其邊際機率 \(p_y(i)\) 由高至低排序;橫軸的離散單元則是依據其對應的最高機率音位 \(y^\ast(j)\) 的縱軸排序位置進行排列。\footnote{如果兩個離散單元 \(j_1\) 和 \(j_2\) 對應到相同的音位 \(y^\ast = y^\ast(j_1) = y^\ast(j_2)\),則依照機率值 \(p_{yz}(y^\ast, j_1)\) 和 \(p_{yz}(y^\ast, j_2)\) 由高到低進行排序,對於多個離散單元的情況以此類推。} 這樣可以在熱圖上呈現出由左上至右下的對應關係。

        為了評估離散表徵是否有捕捉到與音位相關的資訊,我們可以分別從音位與離散單元的兩個角度出發,考慮以下兩個問題:
\begin{enumerate}
    \item 對於每個音位而言,它們所對應的離散單元集中程度如何?
從這個角度出發,可以觀察不同音位的集中程度,進而推論模型是否能夠辨認出該音位的發音特性,藉由給予夠高的一致性將這些語音訊號分類在一起。會不會有某一些音位很難被歸類出來?
    \item 反之,一個離散單元所對應的音位的集中或分散程度如何?
如果一個音框的語音訊號被模型指示為特定的離散單元,該單元作為虛擬標註,能多大程度的對應到人耳感知的音位標註?也就是這些虛擬標註,是否達成音位標註類似的效果,足以把不同語音特徵區分開來。
\end{enumerate}

        這兩個問題的答案可以分別很直觀的從圖中顏色的深淺觀察出來,也正好對應前面所提及的兩個純度指標:
        \begin{enumerate}
            \item 將每個橫列(Row)取最大值相加後的總和即為分群純度
            \item 將每個直行(Column)取最大值相加後的總和則是音位純度
        \end{enumerate}

        從這裡我們可以看到,當分群數量增加時,音位純度可以在每個直行上取到更多的機率值,這也意味著當分群數量與音框數量相同時,音位純度可以達到 100\%,與前面的描述互相吻合。另一方面,當分群數量增加時,每個格子的機率值會因為離散單位數量的增多而被稀釋,而分群純度受到音位數量限制,只能取 41 個 $p_{yz}$ 值的總和,使得單位純度因而明顯降低。

        以上是綜觀整個系統給予虛擬標註時,對應到音位標註的好壞。然而我們可以更進一步的探討各音位與離散單元之間的內部差異,也就是分別探討:
\begin{itemize}
    \item 哪些離散單元比較能集中抓取音位的特徵,不會與其他音位混淆?
    \item 如果一個離散單元被分散的對應到多個音位,那麼這些音位可能是哪幾個?是否存在某些共同特徵?
\end{itemize}

        由於這些問題是以離散單元的角度出發,因此我們仿照前作如 SpeechTokenizer \cite{zhang2024speechtokenizer}、DinoSR \cite{liu2024dinosr} 的作法,將熱圖改以 $p_{y|z}(i|j)$ 呈現,即對每個直行進行標準化得到條件機率,以顯示每個單位對應到哪個音位,探討這種對應如何集中或分散。

\begin{figure}
    \centering
    \includegraphics[width=1\linewidth]{figures/11111111.png}
    \caption{HuBERT 模型、分群數為 50 之\\
$p_{y|z}(i|j)$  條件機率分佈圖}
    \label{fig:hubert-50-givenunit-byprob}
\end{figure}
        從圖 \ref{fig:hubert-50-givenunit-byprob} 中可以更明顯的看出,模型會耗費不少種類的離散單元於編碼非音位的音素標註(尤其是 sil)之上。\jeffcomment{加上 silence ratio?}此外,每個離散單元對於其對應的訊號所對應的音位集中程度有高有低,使得音位純度無法到達 1.00。然而,這邊比較有趣的點是,觀察那些對應音位比較分散的離散單元,我們其實可以發現這些音位彼此之間有很強的關聯性,幾乎與前述的語音分類一致。  % 

        \begin{figure}
            \centering
            \includegraphics[width=1\linewidth]{figures/unit_rank_phn.png}  % figures/unit_perspective.png
            \caption[]{
離散單元對應的前幾高音位示意。圖中的方框、圓圈等形狀}
                                          表示輔音發聲部位,外框顏色則表示清濁音。注意元音都屬於濁音
            \label{fig:unit-to-phn-rankings}
        \end{figure}
        
        這件事可以從熱圖上由左上而右下連線中,不在線上但顏色較深的區塊中觀察出來。但由於直接從熱圖上觀察比較難以呈現,因此我們另外統計出表 \ref{fig:unit-to-phn-rankings},其中展現的是幾個離散單元對應的前五高機率音位,並且用顏色標明各音位所屬的語音學類別。從表中大致可以看出以上描述的趨勢,而且即便不是同一個語音學類別,按照前面講解語音學對音位歸類的另外兩個層面 --- 發音部位和清濁音,還是可以將各離散單元的前幾名之中盡量找出共通點。例如 05 號單元對應的前兩名 /t/ 和 /s/ 雖然並不屬於同一個發聲方式,因而被分成兩個類別,但如果從國際音標表中的「發音部位」來觀察,會發現它們都屬於「齒音」。換言之這些離散單元捕捉到的語音特性是多個面向的,並不僅限於單一的分類方式,而是可以對應到國際音標表上至少兩個維度以上的類型。

\begin{figure}
    \centering
    \includegraphics[width=1\linewidth]{figures/ipa_similarity.png}
    \caption[]{
國際音標表的輔音表格,說明離散單元}
                                                                對語音聲學特徵的捕捉並不僅限單一面向
    \label{fig:ipa-cons-table-sim}
\end{figure}

        透過以上的觀察,因此我們有足夠的理由重新對熱圖的縱軸重新排列,並按照語音學分類進行分組,來觀察這些離散單元是如何指示出音位之間的相似性,區分出同個音位、同類發音,或者如何被混淆為其他類別,而這些類別是否有某些特徵,最後這樣的現象是否只在單一模型出現,抑或是在不同的離散單元系統都會發生。

\begin{figure}
    \centering
    \includegraphics[width=1\linewidth]{figures/hubert-50-givenunit-byphn.png}
    \caption[]{% \medskip % \small
        HuBERT 模型、分群數為 50 之離散單元}
                                                    與音位標註的條件機率,依照語音學分類排序的分佈圖
    \label{fig:hubert-50-givenunit-byphn}
\end{figure}

        這張圖的分組順序是依照韋氏(Wells) \cite{wells_phonetic_2022} 論文中的出現順序排列,而組別內則是清音在上、濁音在下,而同樣清濁音則是以發音位置由前往後排列。除了縱軸上按照音位本身特性分組,依循純度中使用的「代表音位」 $i^\ast$ 概念,我們同樣也對每個離散單元的代表音位排序,並且也依照這些代表音位進行分組觀察。

        為了比較好的刻劃這個在分群內的好壞,我們接下來多算兩個指標:\par
\begin{enumerate}
    \item 語音分類的純度:為了確認每個離散單元如何「將音位至少分到同一語音學類別」的程度,藉由將前面音位純度的式子,但將音位標註改為語音學類別,便可以求得這個數據。
    \item 各發音類別的純度:為了衡量模型對於每個類別內部區分不同音位的能力,比較模型對於不同組別區分音位的難易度,我們可以根據音位的語音學類別,將所有的音框等效分成八份語料後,分別再次統計純度(亦即計算對語音學類別取條件機率後計算純度)。
\end{enumerate}

        說明完以上指標後,我們將展示不同離散表徵模型的所有分析,彼此先進行綜觀比較,此後再針對細部的特徵分析。


}

\section{分析結果}

\subsection{不同語音離散表徵的比較}

  首先是比較不同模型的離散特徵之數據與機率分佈圖:

純度:
{

\begin{figure}
    \centering
    \includegraphics[width=1\linewidth]{figures/000.png}
    \caption{比較表}
    \label{fig:enter-label}
\end{figure}

}
機率分佈圖:
{

\begin{figure}
    \centering
    \includegraphics[width=1\linewidth]{figures/hubert50.png}
    \caption{Hubert}
    \label{fig:enter-label}
\end{figure}
\begin{figure}
    \centering
    \includegraphics[width=1\linewidth]{figures/w2v250.png}
    \caption{w2v2}
    \label{fig:enter-label}
\end{figure}

\begin{figure}
    \centering
    \includegraphics[width=1\linewidth]{figures/cpc50.png}
    \caption{cpc}
    \label{fig:enter-label}
\end{figure}

\begin{figure}
    \centering
    \includegraphics[width=1\linewidth]{figures/logmel50.png}
    \caption{logmel}
    \label{fig:enter-label}
\end{figure}

}

        比較不同的模型的聯合分佈後,我們可以觀察到這些模型之間,確實存在。

        就是說標注跟單元之間的 相關性 高低,比較,從聯合分佈圖上也可以明顯的呈現出來,進而說明這些不同的語音離散表徵之前,捕捉訊號中的發音特徵能力的強弱。從圖中可以看出,HuBERT 跟 CPC 的效果比另外兩種表徵好上不少。

        此外,我們也可以比較同一種語音表徵之下,不同的離散單元分群數之間對於音位特徵捕捉的強弱程度。基於前面表徵純度與相互資訊的考量,這邊固定用 HuBERT 當比較對象。從這三種分群數看來,離散單元數量愈多,愈能夠區分出比較細節的語音類別。例如,如果要從每個 離散單元 的代表音位來觀察,要至少有 100 個群數,才至少有一個離散單元能代表塞擦音。如果分群數量太少,很多細節的發音音位則很容易被放在一起,難以區別出更細節的發音差異。

        再者,我們觀察到模型會消耗一定比例的離散單元去代表非音位的發音,以此我們計算出一個「非音位比例」,也就是等效有幾個 離散單元 被拿去代表這些不是音位的聲音。具體算法是使用 $E [ u | p ] $  總和除以離散單元的數量。

        然後,我們期望看到一個單元系統怎麼有效的去使用這些離散單元,因此我們可以畫一張離散單元音位熵的直方圖(Histogram)來刻劃模型系統在使用這些離散單元的集中程度。如果整體向熵值低處偏,表示模型的各種離散單元所代表的音位都是相對確定的,也就是更好的捕捉到了音位特徵。

        最後,為了觀察各自不同語音學類別內分群的好壞,我們可以算語音學類別的純度,並且再以各個語音學類別區分出來,計算八個類別的各自 條件機率下計算的 純度,以此比較不同表徵、分群數在針對不同語音學類別的特徵補綴效果。

        比較完這些表徵與離散單元數量的各項綜觀統計指標後,我們基於 相互資訊 和 純度 的高低,著重關注於 HuBERT 100 和 50 模型或 CPC,藉由觀察和比較兩個模型共有的規律,往下細部探討該模型所捕捉到的離散單元和音位之間的關係。

\subsection{個案探討}  % 分組討論  %

  首先,承接前面所述的「熱圖上不在斜線上的點」,我們可以觀察每個離散單元所對應的 音位 之間的共同特性。

        接著,我們確認一下對應最紛亂與最集中的幾個 離散單元 的狀況。

        接著,跟隨韋氏(Wells)\cite{wells_phonetic_2022} 之前的作法,我們觀察各類別的 音位 可能各自是以那些離散單元為代表。

        之後,從各個 音位 的熵值來觀察,我們可以發現\textcolor{red}{某某某某某}幾個音位的熵值特高特低,來發現這些音位可能是比較難以捕捉的。而這個現象在 50 模型又可能是怎樣怎樣的。

\subsubsection{切塊出來}

  最後,基於我們已經有對語音學分類,我們可以觀察熱圖上不同語音學類別所切出來的區域的亂度,來觀察各類別的發音特徵捕捉的難易度。例如塞音怎樣怎樣啊……

{


\section{本章總結}

  本章節探討以音框為單位取出的語音離散表徵與對應的音位標註之間的關係,從分析結果中可以看到,HuBERT 模型的離散表徵確實與人類理解的語音單位「音位」之間具有最明顯的相似性,也進一步證明為何 HuBERT 是抽取語音離散表徵時最常使用的模型。

        然而,單一離散表徵僅能代表 10 或 20 毫秒的語音訊號,而音位的長度經常佔據不只一個離散表徵。因此,下一章節將進一步組合多個離散表徵成為符記,分析它們與音位之間的關係。

%   從統計數據出發,我們針對聯合分佈的各個面向,配合了語音學知識的分類進行了細部探討,發現了 hunert 模型怎樣怎樣,而這件事可能在其他的模型之中差不多是 holid 住的。然後,因為 hunert 本身捕捉的各項純度與 MI 明顯較高,以此可以驗證為什麼 HuBERT 的離散單元可以在無文字架構內被當成類似音位或文字的表徵,並進而套用於語音語言模型的訓練上,同時為許多做語音模型解釋性的作品所關注citehao。


}
