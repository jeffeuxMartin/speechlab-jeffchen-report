% ! TEX root = ../../realeasythesis.tex
\chapter{結論與展望} 

\section{研究貢獻與討論}

  本論文旨在細部探討和比較語音基石模型得到的離散表徵與人們理解語音的最小單位 --- 音位之間的關係,藉助語音學知識為語音標註提供的分組方式,拓展純度與相互資訊給予的意義,比較共同或條件機率分佈各自的熵與純度等資訊,細部觀察機器學習到的離散表徵與音位標註之間的相似性與差異。其中,藉由分群演算法所獲得的單一離散單元,以及引入文字處理中分詞演算法重新編碼出的次詞單位 --- 聲學片段,兩者都是將語音訊號離散化的方式。我們比較了離散單元與聲學片段在音位標註之間共通性的變化,探討用不同方式對語音訊號取得符記造成的影響。

        首先,論文第三章介紹了與無文字架構以及語音表徵相關的分析研究,隨後簡介語音學知識中,對於不同音位之間如何按照發音特性分門別類。接著,透過純度與相互資訊定義中統計的共同機率分佈,將機率分佈從離散單元與音位兩個角度切入、比較各自的條件機率分佈特性,觀察不同模型、不同分群參數或不同音位之間是否有特定的集中或分散關係,以及不同離散表徵模型對語音訊號特性歸類的能力。結果可發現,HuBERT 作為目前無文字架構最常用的語音離散表徵模型的理由,很可能來自於它們的音位純度與相互資訊都相對較高,因而更能捕捉到語音中與內容相關的重要資訊,且同樣的趨勢在語音學分類的標註也可以被觀察到。(((寫一下細部的三方向觀察結果)))
從

        其後在論文第四章\textsl{},我們將離散單元以文字處理中的單一詞演算法重新分組編碼為次詞單位序列,以使得不同的離散單元之間可以重新分組成新的符記,並與第三章的結果對照比較,觀察是否在對發音特性的捕捉效果上有所變化。(((寫一下沒什麼好處的結果)))

\section{未來展望}

  希望這些對離散單元與分詞方法應用的嘗試,能幫助我們在訓練任務之前,決定哪種語音基石模型更適合作為離散編碼語音訊號的基礎。接下來,我們期望能針對常見的語音任務,特別是語音辨識和語音翻譯等內容處理相關的任務,比對離散單元促成的實際成效和分析數據之間的關係,並對這些任務中的錯誤案例進行統計和個案探討。

        另外,對於如何結合語音離散單元,除了將其視為文字進行分詞演算法外,我們還可以使用其他方式對離散單元序列進行分組,以達成壓縮序列長度並使其與音位等語音內容更加一致的目標。例如,將此目標形塑為語音分段(Speech Segmentation)任務等,也是未來可以嘗試的離散單元分組方式。

        最後,利用語音學分組的切入點,或許可以在未來分析離散單元或連續語音表徵時,不再僅限於參考音位或文字,還可以從語音學知識提供的相似性資訊出發,為錯誤發音修正等任務提供衡量的依據。
