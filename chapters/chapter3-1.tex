
\section{相關研究}

\subsection{無文字與離散語音表徵}
  
自 HuBERT 帶起的研究之後,
出現了愈來愈多離散表徵相關的研究,
例如 \cite{10097097, abdullah23_interspeech, chang_exploration_2023, liu2024dinosr, zhang2024speechtokenizer, huang2023repcodec} 等等。
它們在提出自己的離散表徵時,
也會採取 HuBERT 的
衡量方式,
% 例如 PNMI 等等的,
來驗證這些離散單元與語音中的內容及人類對語音的詮釋之間
具有一定程度的相關性,
並從資訊理論(Information Theory)的角度,
證明這些離散單元
確實具備區分
不同語音
% 中不同形式的
資訊
的能力
。

\subsection{語音學分析}
  
由於語音處理本身最終是針對人類語音,
因此
有一群研究者通過對人類語音的理解,
將這些知識應用在分析模型如何對語音訊號建構表徵之上。例
如 \cite{deseyssel22_interspeech, wells_phonetic_2022, 10097097, abdullah23_interspeech} 等研究。
% 就韋誠說的妹子那篇嗎?

基於這些作品
對語音離散表徵的興趣和探討,  % 其實是我的興趣?
本論文也先透過過往幾個常用來分析語音表徵的方式,特別是
HuBERT \cite{hsu_hubert_2021-2} 提出的標準進行初步的分析。
% 以下介紹此次分析語音表徵的衡量方式:
