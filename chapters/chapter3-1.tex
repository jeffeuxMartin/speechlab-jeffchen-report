
\section{相關研究}  

\subsection{無文字與離散語音表徵}

  自 HuBERT 帶起的研究之後,出現了愈來愈多離散表徵相關的研究\cite{10097097, abdullah23_interspeech, chang_exploration_2023, liu2024dinosr, zhang2024speechtokenizer, huang2023repcodec} 。它們在提出自己的離散表徵時,也會採取 HuBERT 的衡量方式,來驗證這些離散單元與語音中的內容及人類對語音的詮釋之間,具有一定程度的相關性,並從資訊理論(Information Theory)的角度,證明這些離散單元確實具備區分不同語音資訊的能力。

\subsection{語音學分析}

  由於語音處理本身最終是針對人類語音,因此有一群研究者通過對人類語音的理解,將這些知識應用在分析模型如何對語音訊號建構表徵之上\cite{deseyssel22_interspeech, wells_phonetic_2022, 10097097, abdullah23_interspeech} 。基於這些作品對語音離散表徵的興趣和探討,本論文也先透過過往幾個常用來分析語音表徵的方式,特別是HuBERT \cite{hsu_hubert_2021-2} 提出的標準進行初步的分析。
