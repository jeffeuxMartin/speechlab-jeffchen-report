
\section{相關研究}  %%

\subsection{無文字與離散語音表徵}
  
自 HuBERT 帶起的研究之後,出現了愈來愈多離散表徵相關的研究,例如 \cite{10097097, abdullah23_interspeech, chang_exploration_2023, liu2024dinosr, zhang2024speechtokenizer, huang2023repcodec} 等等。它們在提出自己的離散表徵時,也會採取 HuBERT 的衡量方式,來驗證這些離散單元與語音中的內容及人類對語音的詮釋之間具有一定程度的相關性,並從資訊理論(Information Theory)的角度,證明這些離散單元確實具備區分不同語音資訊的能力。

\subsection{語音學分析}
  
由於語音處理本身最終是針對人類語音,因此有一群研究者通過對人類語音的理解,將這些知識應用在分析模型如何對語音訊號建構表徵之上。例如 \cite{deseyssel22_interspeech, wells_phonetic_2022, 10097097, abdullah23_interspeech} 等研究。

基於這些作品對語音離散表徵的興趣和探討,本論文也先透過過往幾個常用來分析語音表徵的方式,特別是HuBERT \cite{hsu_hubert_2021-2} 提出的標準進行初步的分析。
