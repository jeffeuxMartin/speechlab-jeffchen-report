
        我們這個研究是嘗試探究離散單元與音位之間的關係。

        首先原因是因為語音本身捕捉的是連續的訊號變化,因此我們有了離散單元。然而,人類本身語音學就已經有了離散的符號 --- 文字與音位。於是,我們可以試圖去比較現今語音離散表徵與人類對語音音位歸類的差異來理解這些離散表徵是否有捕捉到類似於人類發音的特性。並且,藉由語音對音位標註之間的分組,我們可以觀察這些離散表徵是否也有相似的分組特性。

        最後,因為人類對音位的感知往往多於單一的語音表徵音框,因此我們可以嘗試借鑑文字處理中的次詞單位,重新編碼語音訊號再次確認這些次詞單位是否類似於人類的發音特性。

\textbf{關鍵字}:語音基石模型、離散單元、語音表徵、語音學
