  
HuBERT \cite{hsu_hubert_2021, hsu_hubert_2021-2} 和 Wav2vec 2.0 \cite{baevski2020wav2vec} 等語音基石模型的成功,不僅在語音任務上達到了前所未有的表現,還促進了語音表徵離散化的發展。由此產生的「無文字(Textless)」架構 \cite{noauthor_textless_2021, lakhotia_generative_2021, lakhotia_generative_2021-1},讓人們在處理語音訊號時,有了連續表徵以外的新選擇。離散形式的表徵可以直接應用文字領域發展的技術,如機器翻譯、生成式模型等,為語音技術帶來新的突破。另一方面,基於離散「符記(Token)」的共同形式,離散語音表徵可以更好的整合文字資料,促成多模態領域的發展。跨模態離散表徵的成功,甚至驅使影像領域也開始發展離散表徵,\jeffcomment{要確認一下「開始」嗎?}如探討唇語的 AV-HuBERT \cite{shi2021learning} 等等,展現了離散表徵在資料處理上的優勢。

此外,除了技術的角度切入,為了探討離散語音表徵成功背後的可能因素,以及它們與語言學對人類語音理解之間的差異,甚至是進一步利用這些技術協助更細緻的探討人類的語音現象。因此,原先在連續語音表徵上的語音學分析,也開始關注離散表徵在多大程度上能描述語音現象,將其列入考量,成為除了連續語音特徵和時頻譜之外的另一個選擇。
