
\section{本章總結}
  
本章節探討以音框為單位取出的語音離散表徵與對應的音位標註之間的關係,從分析結果中可以看到,HuBERT 模型的離散表徵確實與人類理解的語音單位「音位」之間具有最明顯的相似性,也進一步證明
% 以此尋找語音中的
為何 HuBERT 是抽取語音離散表徵時最常使用的模型。

%%% 其實數據有點 bug,/h/ 跟元音不分單雙

然而,單一離散表徵僅能代表 10 或 20 毫秒的語音訊號,而音位的長度經常佔據不只一個離散表徵。因此,下一章節將進一步組合多個離散表徵成為符記,分析它們與音位之間的關係。
