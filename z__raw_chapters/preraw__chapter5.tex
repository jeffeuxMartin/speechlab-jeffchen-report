\chapter{結論與展望}

\section{研究貢獻與討論}
  
本論文的主旨,在於分析語音基石模型的離散表徵,與語音標註之間的純度和相互資訊等數據的相關性,並且透過分詞方法的引入,嘗試將多個離散單元進行結合後,觀察學習到的新符記是否和音位等標註更加一致。 

首先,論文第三章介紹了與無文字架構以及語音表徵相關的分析研究,隨後簡介語音學知識中,對於不同音位之間如何按照發音特性分門別類。有了音位與語音學分類兩種語音標註後,借鑑 HuBERT 提及的純度和相互資訊的分析方式,對離散表徵與語音標註之間,兩者的相關性進行分析與觀察,比對無文字架構中不同語音離散表徵的統計特性。結果可發現,HuBERT 作為目前無文字架構最常用的語音離散表徵模型的理由,很可能來自於它們的音位純度與相互資訊都相對較高,因而更能捕捉到語音中與內容相關的重要資訊,且同樣的趨勢在語音學分類的標註也可以被觀察到。

其後在論文第四章,考慮到音位與離散單元往往是一對多的關係,
藉著嘗試引入自然語言處理常用的分詞方法,
重新對離散單元的序列進行分組,並且比較不同詞表大小對這些分析數據的影響。
考量語音不如英語的文字系統具備明確的空格提示,
本研究採取單一詞作為分詞方法進行
實驗,
並比較不同模型與不同詞表大小對第三章的分析數據是否造成影響。

結果顯示,
藉助加入分詞方法並提供足夠大的詞表,確實能夠讓不同音位的純度以及相互資訊有所提升,
讓多個離散單元之間有機會相互結合、重新分組,更能捕捉語音訊號中的內容資訊。
且在四種語音表徵之間,HuBERT 依然是所有模型中音位純度和相互資訊最高者,還達成了一定的序列長度壓縮比率,相較之下 CPC 模型雖然壓縮比率更低,卻犧牲掉過多語音資訊導致相關數據反而較差。這也解釋了為什麼目前在離散單元相關的研究中,無論是使用單一離散單元,或是使用分詞方法進行長度壓縮等,HuBERT 都仍是最有利於後續語音任務的訓練與應用的模型。

 

\section{未來展望}
  
希望這些對離散單元與分詞方法應用的嘗試,能幫助我們在訓練任務之前,決定哪種語音基石模型更適合作為離散編碼語音訊號的基礎。接下來,我們期望能針對常見的語音任務,特別是語音辨識和語音翻譯等內容處理相關的任務,比對離散單元促成的實際成效和分析數據之間的關係,並對這些任務中的錯誤案例進行統計和個案探討。

另外,對於如何結合語音離散單元,除了將其視為文字進行分詞演算法外,我們還可以使用其他方式對離散單元序列進行分組,以達成壓縮序列長度並使其與音位等語音內容更加一致的目標。例如,將此目標形塑為語音分段(Speech Segmentation)任務等,也是未來可以嘗試的離散單元分組方式。

最後,利用語音學分組的切入點,或許可以在未來分析離散單元或連續語音表徵時,不再僅限於參考音位或文字,還可以從語音學知識提供的相似性資訊出發,為錯誤發音修正等任務提供衡量的依據。
