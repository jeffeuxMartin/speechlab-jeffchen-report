
\chapter{多個語音離散表徵與音位的關係}

\section{動機}

  如前一章所述,一個文字或音位往往對應到上百毫秒的語音訊號,然而單一離散單元所對應的聲音訊號為 10 或 20 毫秒,亦即同一段語音所對應的離散單元數目將比音位或文字多出許多。本章節從自然語言處理中獲取靈感,將\textcolor{red}{分詞演算法(Tokenization)\textcolor{red}{(tokenization 的翻譯需要調整!)}}應用於離散單元序列上,並應用上一章節的分析方法檢驗將多個離散單元所組成之符記。探討分詞後,離散單元是否可以\textcolor{red}{同時擁有無文字(Textless)}\cite{lakhotia_generative_2021, lakhotia_generative_2021-1, noauthor_textless_2021} 的特性,且更接近音位的序列,成為更好的語音表徵。

\section{相關研究} 

  在無文字架構被提出後的約兩年後,藉分詞方法組合離散單元的研究逐步出現。最初提出「聲學片段(Acoustic Piece)」的是任氏(Ren)等人 \cite{ren_speech_2022}\citetag{1-22A4-Pretrain-ap}),該論文比對離散單元序列及對應的文字轉寫,從中觀察到許多相似的規律重複出現,而且不限於單一語者。受此啟發,本論文首先將離散單元使用句子片段(SentencePiece) \cite{kudo_sentencepiece_2018} 分詞,獲得新的符記 --- 「聲學片段」,並用於語音辨識的預訓練上。

        不久,由吳氏(Wu)提出的 Wav2seq \cite{wu_wav2seq_2023}\citetag{2-22A5-wav2seq}論文中,考量文字與語音的序列長度差異,並基於離散單元和音位的關聯性,將離散單元視為字符(Character),嘗試將這些字符透過分詞方法組成「虛擬語言(Pseudo-language\footnote{偽語言對應之離散單元被視為「虛擬文字(Pseudo-text)」})」,來幫助語音到文字的模型。因為解碼器在實際應用時需要生成的序列多是文字的符記 --- 次詞單位(Subword Unit),因此該篇研究旨在讓模型在預訓練後可以快速適應下游任務。與前一篇呼應,「聲學片段」對語音預訓練的效果在\cite{10096788}\citetag{3-23A-coarser-grain}中被探討,此後聲學片段更被應用於縮短資料序列長度\cite{chang_exploration_2023}\citetag{4-23B-Exploration of Efficient End-to-End ASR using Discretized Input from Self-Supervised Learning} 、語音生成\cite{shen2024acoustic}\citetag{5-24A-speech-gen},或學習更穩健(Robust)的語音表徵\cite{chang2023r}\citetag{6-23B-rspin-acousticpiece}。

        近期,張氏(Chang)等人\cite{chang_exploring_2024}\citetag{7-23B-shinji-hsiuhsuan}將以分詞方法處理離散單元的流程(Pipeline)納入 ESPNet 套件 \cite{watanabe2018espnet} 中,並在語音辨識、語音翻譯等任務中獲得了超越以往的表現,進一步證明了這個方法的效果。

\section{分詞方法}

  在以文字為主體的自然語言處理中,文字文本除了以單詞(Word)或字元(Character)為處理單位,更常見的作法是透過分詞演算法(Tokenization)將文本分段,以「次詞單位」構成詞彙表來重新編碼文本,用於文字模型的訓練與推理。

        分詞方法的優點一般包含:

\begin{enumerate}
    \item 固定詞彙表大小,避免未登錄詞(Out-of-vocabulary,OOV)
    \item 縮短資料序列的長度,提升訓練和推論的效率。
    \item 分解單詞,捕捉更細緻的語意關係,模擬如英語中的字首(Prefix)、字尾(Suffix)等等具有特定意義的文字組合。
\end{enumerate}

\subsection{常見演算法}

  以下介紹幾種常見的分詞方法:

\paragraph{位元組對編碼(Byte Pair Encoding,BPE)} \hfill \break
%
  位元組對編碼 \cite{10.5555/177910.177914, sennrich_neural_2016} 是一種常用的分詞方法,最初來自資料壓縮技術 \cite{10.5555/177910.177914},後來被引入到自然語言處理領域,用以處理機器翻譯問題 \cite{sennrich_neural_2016} 。
該演算法從字元開始,根據詞彙表中各個次詞單位的頻率,反覆合併常見的字元成為新的次詞單位,直到達到預定的詞彙表大小。

\paragraph{單詞片段(WordPiece)} \hfill \break
%
  WordPiece \cite{wu2016google} 演算法由 Google 用以訓練機器翻譯系統,並在 BERT \cite{devlin_bert_2019} 模型中被使用而廣為人知。與位元組對編碼相似,同樣是透過反覆合併的策略,但合併的依據改以機率模型取代出現頻率。

\paragraph{單一詞語言模型(Unigram Language Model)} \hfill \break
%
  單一詞語言模型 \cite{kudo2018subword} 是基於語言模型的分詞方法,以機率分佈選擇次詞單位,並以最大化輸入文本的機率來為文本分段。

\subsection{句子片段(SentencePiece)套件}

  SentencePiece \cite{kudo_sentencepiece_2018}
是由 Google 開發的分詞套件,實作了前述的位元組對編碼和單一詞演算法。其優勢在於可應用於不同語言,尤其用於處理中文、日文等不使用空格分隔單詞的語言文本時,此套件大大的簡化了前處理的流程。

\section{衡量方式}

  本章節沿用上一章節 LibriSpeech 資料集的 train-clean-100 訓練子集,以及相同的分析數據以進行比對。由於與上一章節的差異僅在分詞方法的引入,因此數據結果將多紀錄分詞前後序列長度的變化;其他操控變因包含分詞方法與詞表大小。考慮到語音訊號本身不如英語等文字,在書寫時就已經具備空格分隔單詞,因此以下分析結果皆採用 SentencePiece 套件中實作之單一詞演算法為分詞方法,並比較詞表大小 500、1000、8000、10000、20000 五種設定的結果差異。(空間不足時則僅呈現 500、1000、10000 三種設定的趨勢。)

\section{分析結果} \newcommand{\jefftablesep}{\vspace{0.5cm}}\renewcommand{\arraystretch}{0.7} % 調整行高

  以下數據中的「長度壓縮比率」係指透過分詞方法後,每一個句子的「單詞」數與原先離散單元的數量相比之比值。由於長度壓縮比率是針對離散單元分詞所得到,與標註無關,因此只在音位分析的表格上呈現。

\subsection{基於各自音位的分析}

  首先,為了比較不同詞表大小對於純度、相互資訊等數據的影響,先分別固定語音模型為 HuBERT 和 Wav2vec 2.0,在離散單元的分群數量為 50、100、200 三種設定下,觀察詞表大小造成的變化。由 HuBERT (表 \ref{tab:hubert-phn-results})和 Wav2vec 2.0(表 \ref{tab:w2v2-phn-results})的數據比較可觀察到,詞表大小上升除了使得音位純度提高以外,相互資訊也是隨之提高的,可以發現使用分詞方法並給予足夠大的詞表,對於找出語音中的資訊確實有所幫助。

        接著從另一個角度切入,比較同樣都是離散單元分群數為 200 的條件下,不同語音基石模型的分析數據。由表 \ref{tab:ch4-models-phn} 可以發現,HuBERT 模型在音位純度與相互資訊勝過其他模型,這個結論與上一章節是一致的。

        有趣的是,觀察長度壓縮比率可以發現,CPC 模型在分詞演算法的引入後,能夠使序列變得最短,但同時在音位純度與相互資訊上也有所犧牲;而 HuBERT 雖然在這些分析數據上高過其他三者,卻同時達成了比 Wav2vec 2.0 和 LogMel 更好的壓縮比率。因此綜合看來,這很可能是目前使用語音離散單元進行研究時,HuBERT 模型仍然是領域內首選的緣由。

                
\begin{table}[!htbp]
    \centering
    \begin{subtable}[t]{\textwidth}
        \centering
        \begin{tabular}{|c|c|c|c|c|c|c|} \hline 
                詞表大小  & 音位純度 & 分群純度 & 音位熵 & 離散單元熵 &    PNMI & 長度壓縮比率 \\ \hline 
50 (未分詞)& 0.5256& 0.3382& 3.3152& 3.8681& 0.4993&1.0000\\ \hline 
                   500  &   0.5574   &  0.0829 &   3.3152  &  6.0282 & 0.5357 & 0.3486  \\ \hline %%  1.7758       
                  1000  &   0.5744   &  0.0556 &   3.3152  &  6.6594 & 0.5466 & 0.2992  \\ \hline %%  1.8120       
                  8000  &   0.5957   &  0.0257 &   3.3152  &  8.5192 & 0.5729 & 0.2074  \\ \hline %%  1.8993       
                 10000  &   0.5955   &  0.0238 &   3.3152  &  8.7207 & 0.5750 & 0.2007  \\ \hline %%  1.9063       
                 20000  &   0.5921   &  0.0182 &   3.3152  &  9.3527 & 0.5820 & 0.1819  \\ \hline %%  1.9293       
        \end{tabular}
\caption{群數 = 50}
        \label{tab:ch4-hubert-phn-clu050}
    \end{subtable}        

    \jefftablesep        

    \begin{subtable}[t]{\textwidth}
        \centering
        \begin{tabular}{|c|c|c|c|c|c|c|} \hline 
                詞表大小  & 音位純度 & 分群純度 & 音位熵 & 離散單元熵 &    PNMI & 長度壓縮比率 \\ \hline 
100 (未分詞)& 0.6097& 0.2553& 3.3152& 4.5704& 0.5786&1.0000\\ \hline 
                   500  &   0.6260   &  0.0972 &   3.3152  &  6.0655 & 0.5990 & 0.4432  \\ \hline %%  1.9858       
                  1000  &   0.6372   &  0.0631 &   3.3152  &  6.7181 & 0.6089 & 0.3666  \\ \hline %%  2.0186       
                  8000  &   0.6536   &  0.0237 &   3.3152  &  8.5954 & 0.6308 & 0.2444  \\ \hline %%  2.0912       
                 10000  &   0.6527   &  0.0219 &   3.3152  &  8.7938 & 0.6324 & 0.2357  \\ \hline %%  2.0965       
                 20000  &   0.6490   &  0.0173 &   3.3152  &  9.4123 & 0.6378 & 0.2123  \\ \hline %%  2.1145       
        \end{tabular}
\caption{群數 = 100}
        \label{tab:ch4-hubert-phn-clu100}
    \end{subtable}        

    \jefftablesep        

    \begin{subtable}[t]{\textwidth}
        \centering
        \begin{tabular}{|c|c|c|c|c|c|c|} \hline 
                詞表大小  & 音位純度 & 分群純度 & 音位熵 & 離散單元熵 &    PNMI & 長度壓縮比率 \\ \hline 
200 (未分詞)& 0.6474& 0.1644& 3.3152& 5.2681& 0.6289&1.0000\\ \hline 
                   500  &   0.6471   &  0.0930 &   3.3152  &  6.0986 & 0.6314 & 0.5995  \\ \hline %%  2.0934       
                  1000  &   0.6540   &  0.0558 &   3.3152  &  6.7786 & 0.6382 & 0.4609  \\ \hline %%  2.1156       
                  8000  &   0.6690   &  0.0208 &   3.3152  &  8.6544 & 0.6567 & 0.2874  \\ \hline %%  2.1769       
                 10000  &   0.6693   &  0.0189 &   3.3152  &  8.8535 & 0.6584 & 0.2764  \\ \hline %%  2.1826       
                 20000  &   0.6684   &  0.0144 &   3.3152  &  9.4737 & 0.6636 & 0.2467  \\ \hline %%  2.1999      
        \end{tabular}
\caption{群數 = 200}
        \label{tab:ch4-hubert-phn-clu200}
    \end{subtable}        

\caption{HuBERT 模型在不同詞表大小時的音位分析數據}
    \label{tab:hubert-phn-results}
\end{table}



\begin{table}[!htbp]
    \centering
    \begin{subtable}[t]{\textwidth}
        \centering
        \begin{tabular}{|c|c|c|c|c|c|c|} \hline 
                詞表大小  & 音位純度 & 分群純度 & 音位熵 & 離散單元熵 &    PNMI & 長度壓縮比率 \\ \hline 
 50 (未分詞)&  0.4006 &   0.2676 & 3.3152 &     3.8215 & 0.3706&1.0000\\ \hline 
                  500  &   0.4295&     0.0594    &3.3152 &   6.0328  &       0.4027 &0.3943 \\ \hline %%  
                 1000  &   0.4411&     0.0456    &3.3152 &   6.6250  &       0.4132 &0.3448 \\ \hline %%  
                 8000  &   0.4676&     0.0239    &3.3152 &   8.4954  &       0.4438 &0.2429 \\ \hline %%  
                10000  &   0.4705&     0.0227    &3.3152 &   8.6966  &       0.4477 &0.2352 \\ \hline %%  
                20000  &   0.4753&     0.0185    &3.3152 &   9.3471  &       0.4587 &0.2132 \\ \hline %%  
        \end{tabular}
\caption{群數 = 50}
        \label{tab:ch4-w2v2-phn-clu050}
    \end{subtable}        

    \jefftablesep        

    \begin{subtable}[t]{\textwidth}
        \centering
        \begin{tabular}{|c|c|c|c|c|c|c|} \hline 
                詞表大小  & 音位純度 & 分群純度 & 音位熵 & 離散單元熵 &    PNMI & 長度壓縮比率 \\ \hline 
 100 (未分詞)&0.4877 &   0.2118 & 3.3152 &     4.5284 & 0.4596&1.0000\\ \hline 
                  500  &   0.5016&     0.0697    &3.3152 &   6.0769  &       0.4784 &0.4926 \\ \hline %%  
                 1000  &   0.5108&     0.0444    &3.3152 &   6.7265  &       0.4867 &0.4160 \\ \hline %%  
                 8000  &   0.5427&     0.0215    &3.3152 &   8.5576  &       0.5180 &0.2895 \\ \hline %%  
                10000  &   0.5451&     0.0203    &3.3152 &   8.7602  &       0.5218 &0.2797 \\ \hline %%  
                20000  &   0.5531&     0.0162    &3.3152 &   9.4070  &       0.5345 &0.2526 \\ \hline %%  
        \end{tabular}
\caption{群數 = 100}
        \label{tab:ch4-w2v2-phn-clu100}
    \end{subtable}        

    \jefftablesep        

    \begin{subtable}[t]{\textwidth}
        \centering
        \begin{tabular}{|c|c|c|c|c|c|c|} \hline 
                詞表大小  & 音位純度 & 分群純度 & 音位熵 & 離散單元熵 &    PNMI & 長度壓縮比率 \\ \hline 
 200 (未分詞)&  0.5427 &   0.1467 & 3.3152 &     5.2173 & 0.5188 &1.0000\\ \hline 
                  500  &   0.5481&     0.0846    &3.3152 &   6.0756  &       0.5276 &0.6418 \\ \hline %%  
                 1000  &   0.5549&     0.0509    &3.3152 &   6.7483  &       0.5347 &0.5178 \\ \hline %%  
                 8000  &   0.5803&     0.0180    &3.3152 &   8.6065  &       0.5632 &0.3498 \\ \hline %%  
                10000  &   0.5828&     0.0166    &3.3152 &   8.8136  &       0.5668 &0.3373 \\ \hline %%  
                20000  &   0.5917&     0.0129    &3.3152 &   9.4561  &       0.5794 &0.3026  \\ \hline %%  
        \end{tabular}
\caption{群數 = 200}
        \label{tab:ch4-w2v2-phn-clu200}
    \end{subtable}        

\caption{Wav2vec 2.0 模型在不同詞表大小時的音位分析數據}
    \label{tab:w2v2-phn-results}
\end{table}




\begin{table}[!htbp]
    \centering
    \begin{subtable}[t]{\textwidth}
        \centering
        \begin{tabular}{|c|c|c|c|c|c|c|} \hline 
                詞表大小  & 音位純度 & 分群純度 & 音位熵 & 離散單元熵 &    PNMI & 長度壓縮比率 \\ \hline 
200 (未分詞)&   0.6474 &   0.1644 & 3.3152 &     5.2681 & 0.6289 &1.0000\\ \hline 
                    500   & 0.6471   & 0.0930   & 3.3152 &  6.0986    &  0.6314 &  0.5995  \\ \hline %%   2.0934
                   1000   & 0.6540   & 0.0558   & 3.3152 &  6.7786    &  0.6382 &  0.4609  \\ \hline %%   2.1156
                  10000   & 0.6693   & 0.0189   & 3.3152 &  8.8535    &  0.6584 &  0.2764  \\ \hline %%   2.1826
        \end{tabular}
\caption{HuBERT}
        \label{tab:ch4-phn-model-hubert}
    \end{subtable}        

    \jefftablesep        

    \begin{subtable}[t]{\textwidth}
        \centering
        \begin{tabular}{|c|c|c|c|c|c|c|} \hline 
                詞表大小  & 音位純度 & 分群純度 & 音位熵 & 離散單元熵 &    PNMI & 長度壓縮比率 \\ \hline 
200 (未分詞)&   0.5427 &   0.1467 & 3.3152 &     5.2173 & 0.5188 &1.0000\\ \hline 
                    500   & 0.5481   & 0.0846   & 3.3152 &  6.0756    &  0.5276 &  0.6418  \\ \hline %%   1.7491
                   1000   & 0.5549   & 0.0509   & 3.3152 &  6.7483    &  0.5347 &  0.5178  \\ \hline %%   1.7727
                  10000   & 0.5828   & 0.0166   & 3.3152 &  8.8136    &  0.5668 &  0.3373  \\ \hline %%   1.8791
        \end{tabular}
\caption{Wav2vec 2.0}
        \label{tab:ch4-phn-model-w2v2}
    \end{subtable}        

    
    \jefftablesep        

    \begin{subtable}[t]{\textwidth}
        \centering
        \begin{tabular}{|c|c|c|c|c|c|c|} \hline 
                詞表大小  & 音位純度 & 分群純度 & 音位熵 & 離散單元熵 &    PNMI & 長度壓縮比率 \\ \hline 
200 (未分詞)&   0.6098 &   0.1789 & 3.3146 &     5.1885 & 0.5882 &1.0000\\ \hline 
                    500   & 0.6116   & 0.1048   & 3.3146 &  6.0343    &  0.5936 &  0.4951  \\ \hline %%   1.9677
                   1000   & 0.6134   & 0.0618   & 3.3146 &  6.7245    &  0.5979 &  0.3523  \\ \hline %%   1.9818
                  10000   & 0.6198   & 0.0171   & 3.3146 &  8.8593    &  0.6118 &  0.2061  \\ \hline %%   2.0277
        \end{tabular}
\caption{CPC}
        \label{tab:ch4-phn-model-cpc}
    \end{subtable}        

    \jefftablesep        

    \begin{subtable}[t]{\textwidth}
        \centering
        \begin{tabular}{|c|c|c|c|c|c|c|} \hline 
                詞表大小  & 音位純度 & 分群純度 & 音位熵 & 離散單元熵 &    PNMI & 長度壓縮比率 \\ \hline 
200 (未分詞)&   0.3474 &   0.0569 & 3.3158 &     5.2322 & 0.2955 &1.0000\\ \hline 
                    500   & 0.3515   & 0.0403   & 3.3158 &  6.1035    &  0.3066 &  0.6304  \\ \hline %%   1.0166
                   1000   & 0.3547   & 0.0228   & 3.3158 &  6.7602    &  0.3129 &  0.5185  \\ \hline %%   1.0376
                  10000   & 0.3699   & 0.0121   & 3.3158 &  8.7579    &  0.3381 &  0.3427  \\ \hline %%   1.1211
        \end{tabular}
\caption{LogMel}
        \label{tab:ch4-phn-model-logmel}
    \end{subtable}        

\caption{固定離散單元群數皆為 200,不同基石模型的音位分析數據}
    \label{tab:ch4-models-phn}
\end{table}


\subsection{基於語音學分類的分析}

  最後觀察將語音標註換成語音學分類的結果,一樣可以從表 \ref{tab:hubert-pcls-results}、\ref{tab:w2v2-pcls-results} 和 \ref{tab:ch4-models-pcls} 觀察到與上一小節相同的趨勢。

                
        \begin{table}[!htbp]
            \centering
            \begin{subtable}[t]{\textwidth}
                \centering
                \begin{tabular}{|c|c|c|c|c|c|} \hline 
                        詞表大小  & 標註純度 & 分群純度 & 標註熵 & 離散單元熵 &     NMI   \\ \hline 
                       50 (未分詞)&   0.7466  &  0.1422 & 1.7530 &   3.8681 &  0.5742 \\ \hline 
                           500    &  0.7510  &  0.0345  & 1.7530 &  6.0282  &     0.5789  \\ \hline 
                          1000    &  0.7492  &  0.0225  & 1.7530 &  6.6594  &     0.5756  \\ \hline 
                          8000    &  0.7288  &  0.0116  & 1.7530 &  8.5192  &     0.5630  \\ \hline 
                         10000    &  0.7248  &  0.0110  & 1.7530 &  8.7207  &     0.5606  \\ \hline 
                         20000    &  0.7109  &  0.0089  & 1.7530 &  9.3527  &     0.5537  \\ \hline 
                \end{tabular}
\caption{群數 = 50}
                \label{tab:ch4-hubert-pcls-clu050}
            \end{subtable}        

            \jefftablesep        

            \begin{subtable}[t]{\textwidth}
                \centering
                \begin{tabular}{|c|c|c|c|c|c|} \hline 
                        詞表大小  & 標註純度 & 分群純度 & 標註熵 & 離散單元熵 &     NMI   \\ \hline 
 100 (未分詞)&              0.7804 &   0.0856 &         1.7530 &     4.5704 &  0.6148\\ \hline 
                           500    &  0.7787  &  0.0300  & 1.7530 &  6.0655  &     0.6178  \\ \hline 
                          1000    &  0.7751  &  0.0210  & 1.7530 &  6.7181  &     0.6176  \\ \hline 
                          8000    &  0.7507  &  0.0095  & 1.7530 &  8.5954  &     0.6030  \\ \hline 
                         10000    &  0.7468  &  0.0087  & 1.7530 &  8.7938  &     0.6008  \\ \hline 
                         20000    &  0.7347  &  0.0072  & 1.7530 &  9.4123  &     0.5943  \\ \hline 
                \end{tabular}
\caption{群數 = 100}
                \label{tab:ch4-hubert-pcls-clu100}
            \end{subtable}        

            \jefftablesep        

            \begin{subtable}[t]{\textwidth}
                \centering
                \begin{tabular}{|c|c|c|c|c|c|} \hline 
                        詞表大小  & 標註純度 & 分群純度 & 標註熵 & 離散單元熵 &     NMI   \\ \hline 
 200 (未分詞)&              0.8004 &   0.0464 &         1.7530 &     5.2681 &  0.6563\\ \hline 
                           500    &  0.7875  &  0.0269  & 1.7530 &  6.0986  &     0.6469  \\ \hline 
                          1000    &  0.7788  &  0.0170  & 1.7530 &  6.7786  &     0.6385  \\ \hline 
                          8000    &  0.7625  &  0.0071  & 1.7530 &  8.6544  &     0.6266  \\ \hline 
                         10000    &  0.7601  &  0.0064  & 1.7530 &  8.8535  &     0.6251  \\ \hline 
                         20000    &  0.7514  &  0.0051  & 1.7530 &  9.4737  &     0.6200   \\ \hline
                \end{tabular}
\caption{群數 = 200}
                \label{tab:ch4-hubert-pcls-clu200}
            \end{subtable}        

\caption{HuBERT 模型在不同詞表大小時的語音學類別分析數據}
            \label{tab:hubert-pcls-results}
        \end{table}

\begin{table}[!htbp]
    \centering
    \begin{subtable}[t]{\textwidth}
        \centering
        \begin{tabular}{|c|c|c|c|c|c|} \hline 
                詞表大小 & 標註純度 & 分群純度 & 標註熵 & 離散單元熵 &     NMI \\ \hline 
 50 (未分詞)&      0.6913 &   0.1570 &         1.7530 &     3.8215 &  0.4682\\ \hline 
                  500  &  0.7018 &    0.0339 &   1.7530 &   6.0328 &       0.4873  \\ \hline 
                 1000  &  0.7058 &    0.0273 &   1.7530 &   6.6250 &       0.4913  \\ \hline 
                 8000  &  0.7041 &    0.0158 &   1.7530 &   8.4954 &       0.4956  \\ \hline 
                10000  &  0.7033 &    0.0152 &   1.7530 &   8.6966 &       0.4963  \\ \hline 
                20000  &  0.6961 &    0.0128 &   1.7530 &   9.3471 &       0.4954  \\ \hline 
        \end{tabular}
\caption{群數 = 50}
        \label{tab:ch4-w2v2-pcls-clu050}
    \end{subtable}        

    \jefftablesep        

    \begin{subtable}[t]{\textwidth}
        \centering
        \begin{tabular}{|c|c|c|c|c|c|} \hline 
                詞表大小 & 標註純度 & 分群純度 & 標註熵 & 離散單元熵 &     NMI \\ \hline 
 100 (未分詞)&     0.7219 &   0.0889 &         1.7530 &     4.5284 &  0.5252 \\ \hline 
                  500  &  0.7142 &    0.0288 &   1.7530 &   6.0769 &       0.5297  \\ \hline 
                 1000  &  0.7132 &    0.0197 &   1.7530 &   6.7265 &       0.5294  \\ \hline 
                 8000  &  0.7173 &    0.0108 &   1.7530 &   8.5576 &       0.5352  \\ \hline 
                10000  &  0.7171 &    0.0105 &   1.7530 &   8.7602 &       0.5360  \\ \hline 
                20000  &  0.7149 &    0.0089 &   1.7530 &   9.4070 &       0.5379  \\ \hline 
        \end{tabular}
\caption{群數 = 100}
        \label{tab:ch4-w2v2-pcls-clu100}
    \end{subtable}        

    \jefftablesep        

    \begin{subtable}[t]{\textwidth}
        \centering
        \begin{tabular}{|c|c|c|c|c|c|} \hline 
                詞表大小 & 標註純度 & 分群純度 & 標註熵 & 離散單元熵 &     NMI \\ \hline 
  200 (未分詞)&     0.7490 &   0.0527 &         1.7530 &     5.2173 &  0.5671  \\ \hline 
                  500  &  0.7465 &    0.0310 &   1.7530 &   6.0756 &       0.5685  \\ \hline 
                 1000  &  0.7451 &    0.0199 &   1.7530 &   6.7483 &       0.5692  \\ \hline 
                 8000  &  0.7405 &    0.0071 &   1.7530 &   8.6065 &       0.5725  \\ \hline 
                10000  &  0.7399 &    0.0066 &   1.7530 &   8.8136 &       0.5729  \\ \hline 
                20000  &  0.7391 &    0.0055 &   1.7530 &   9.4561 &       0.5757   \\ \hline
        \end{tabular}
\caption{群數 = 200}
        \label{tab:ch4-w2v2-pcls-clu200}
    \end{subtable}        

\caption{Wav2vec 2.0 模型在不同詞表大小時的語音學類別分析數據}
    \label{tab:w2v2-pcls-results}
\end{table}


        \begin{table}[!htbp]
            \centering
            \begin{subtable}[t]{\textwidth}
                \centering
                \begin{tabular}{|c|c|c|c|c|c|} \hline 
                        詞表大小  & 標註純度 & 分群純度 & 標註熵 & 離散單元熵 &     NMI  \\ \hline 
  200 (未分詞)&         0.8004 &   0.0464 &         1.7530 &     5.2681 &  0.6563\\ \hline 
                            500 &  0.7875 &    0.0269 &   1.7530   & 6.0986 &    0.6469 \\ \hline %%   2.0934
                           1000 &  0.7788 &    0.0170 &   1.7530   & 6.7786 &    0.6385 \\ \hline %%   2.1156
                          10000 &  0.7601 &    0.0064 &   1.7530   & 8.8535 &    0.6251 \\ \hline %%   2.1826
                \end{tabular}
\caption{HuBERT}
                \label{tab:ch4-pcls-model-hubert}
            \end{subtable}        

            \jefftablesep        

            \begin{subtable}[t]{\textwidth}
                \centering
                \begin{tabular}{|c|c|c|c|c|c|} \hline 
                        詞表大小  & 標註純度 & 分群純度 & 標註熵 & 離散單元熵 &     NMI  \\ \hline 
  200 (未分詞)&         0.7490 &   0.0527 &         1.7530 &     5.2173 &  0.5671 \\ \hline 
                            500 &  0.7465 &    0.0310 &   1.7530   & 6.0756 &    0.5685 \\ \hline %%   1.7491
                           1000 &  0.7451 &    0.0199 &   1.7530   & 6.7483 &    0.5692 \\ \hline %%   1.7727
                          10000 &  0.7399 &    0.0066 &   1.7530   & 8.8136 &    0.5729 \\ \hline %%   1.8791
                \end{tabular}
\caption{Wav2vec 2.0}
                \label{tab:ch4-pcls-model-w2v2}
            \end{subtable}        

            
            \jefftablesep        

            \begin{subtable}[t]{\textwidth}
                \centering
                \begin{tabular}{|c|c|c|c|c|c|} \hline 
                        詞表大小  & 標註純度 & 分群純度 & 標註熵 & 離散單元熵 &     NMI  \\ \hline 
 200 (未分詞)&         0.7947 &   0.0644 &         1.7530 &     5.1885 &  0.6345 \\ \hline 
                            500 &  0.7925 &    0.0369 &   1.7530   & 6.0343 &    0.6368 \\ \hline %%   1.9677
                           1000 &  0.7882 &    0.0223 &   1.7530   & 6.7245 &    0.6349 \\ \hline %%   1.9818
                          10000 &  0.7609 &    0.0096 &   1.7530   & 8.8593 &    0.6175 \\ \hline %%   2.0277
                \end{tabular}
\caption{CPC}
                \label{tab:ch4-pcls-model-cpc}
            \end{subtable}        

            \jefftablesep        

            \begin{subtable}[t]{\textwidth}
                \centering
                \begin{tabular}{|c|c|c|c|c|c|} \hline 
                詞表大小 & 標註純度 & 分群純度 & 標註熵 & 離散單元熵 &     NMI \\ \hline 
  200 (未分詞)&         0.6107 &   0.0335 &         1.7530 &     5.2322 &  0.3652 \\ \hline 
                            500 &  0.6156 &    0.0247 &   1.7530   & 6.1035 &    0.3801 \\ \hline %%   1.0166
                           1000 &  0.6189 &    0.0137 &   1.7530   & 6.7602 &    0.3875 \\ \hline %%   1.0376
                          10000 &  0.6322 &    0.0096 &   1.7530   & 8.7579 &    0.4085 \\ \hline %%   1.1211
                \end{tabular}
\caption{LogMel}
                \label{tab:ch4-pcls-model-logmel}
            \end{subtable}        

\caption{固定離散單元群數皆為 200,不同基石模型的語音學類別分析數據}
            \label{tab:ch4-models-pcls}
        \end{table}
        


\section{本章總結}

  藉由分詞演算法的引入,我們可以發現在序列長度相對縮短的前提下,音位的純度卻也獲得了提升,足以證明分詞演算法的引入,可以幫助離散單元考量多於一個音框的語音資訊,建構於精細的音框之上,找出更接近人類解讀語音最小單位資訊。期望以此發現,可以使得語音語言模型建立時,模型在處理語音語料庫時,能夠以更接近文字的序列長度與資訊進行訓練,獲得更接近文字模型的效果。
