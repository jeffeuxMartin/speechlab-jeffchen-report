
\section{本章總結}

  藉由分詞演算法的引入,我們可以發現在序列長度相對縮短的前提下,音位的純度卻也獲得了提升,足以證明分詞演算法的引入,可以幫助離散單元考量多於一個音框的語音資訊,建構於精細的音框之上,找出更接近人類解讀語音最小單位資訊。期望以此發現,可以使得語音語言模型建立時,模型在處理語音語料庫時,能夠以更接近文字的序列長度與資訊進行訓練,獲得更接近文字模型的效果。
