
        Our study is an attempt to explore the relationship between discrete units and phonemes.

         The first reason is that speech itself captures continuous signal changes. So we have discrete units. However, human phonetics already has discrete symbols - words and phonemes. Therefore, we can try to compare the differences between today's discrete representations of speech and human classification of speech phonemes to understand whether these discrete representations capture characteristics similar to human pronunciation. Moreover, by grouping phonetic to phoneme labels, we can observe whether these discrete representations also have similar grouping properties.

        Finally, because humans often perceive more phonemes than a single phonetic representation of a sound frame, we can try to learn from sub-word units in word processing and re-encode the speech signal to confirm again whether these sub-word units are similar to human pronunciation characteristics.

\textbf{Keywords}: Speech cornerstone model, discrete unit, speech representation, phonetics
